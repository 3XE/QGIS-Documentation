%  !TeX  root  =  user_guide.tex  
\mainmatter
\pagestyle{scrheadings}
\addchap{Avant-propos}\label{label_forward}

Bienvenue dans le monde merveilleux des Systèmes d'Informations Géographiques 
(SIG) ! Quantum GIS est un SIG libre qui a débuté en mai 2002 et s'est établi en 
tant que projet en juin 2002 sur SourceForge. Nous avons travaillé dur pour faire 
de ce logiciel SIG un choix accessible et viable pour toute personne ayant un 
ordinateur  (qui sont traditionnellement des logiciels propriétaires assez 
coûteux). \qg est utilisable sur la majorité des Unix, Mac OS X et Windows. 
\qg utilise la bibliothèque logicielle Qt (\url{http://qt.nokia.com}) et le 
langage C++, ce qui ce traduit par une interface graphique simple et réactive.

\qg se veut simple à utiliser, fournissant des fonctionnalités courantes. Le but 
initial était de fournir un visionneur de données SIG. \qg a, depuis, atteint un 
stade dans son évolution où beaucoup y recourent pour leurs besoins quotidiens. 
\qg supporte un grand nombre de formats raster et vecteur, avec le support de 
nouveaux formats facilité par l'architecture des modules d'extension.

\qg est distribué sous la licence GNU GPL (General Public License). Ceci signifie 
que vous pouvez étudier et modifier le code source, tout en ayant la garantie 
d'avoir accès à un programme SIG non onéreux et librement modifiable. Vous devez 
avoir reçu une copie complète de la licence avec votre exemplaire de \qg, que 
vous pouvez également trouver dans l'Annexe \ref{gpl_appendix}.

\begin{Tip}\caption{\textsc{Documentation à jour}}\index{documentation}
La dernière version de ce document est disponible sur 
\url{http://download.osgeo.org/qgis/doc/manual/}, ou dans la section 
documentation du site de \qg \url{http://qgis.osgeo.org/documentation}
\end{Tip}

\addsec{Fonctionnalités}\label{label_majfeat}

\qg offre beaucoup d'outils SIG standards par défaut et via les extensions de 
multiples contributeurs. Voici un bref résumé en six catégories qui vous donnera 
un premier aperçu.

\minisec{Visualiser des données}

Vous pouvez afficher et superposer des couches de données rasters et vecteurs 
dans différents formats et projections \footnote{\qg ne proposant actuellement 
de projection à la volée que pour les données de type vecteur, les données de 
type raster doivent être dans la même projection pour pouvoir être associées 
entre elles.} sans avoir à faire de conversion dans un format commun. Les 
formats supportés incluent :

\begin{itemize}[label=--]
\item les tables spatiales de \ppg et SpatiaLite, les formats vecteurs gérés par la 
bibliothèque OGR installée, ce qui inclut les fichiers de format ESRI (shapefiles),
 MapInfo, STDS, GML et beaucoup d'autres,
\item les formats raster supportés par la bibliothèque GDAL (Geospatial Data 
Abstraction Library) tels que GeoTiff, Erdas img., ArcInfo ascii grid, JPEG, 
PNG et beaucoup d'autres,
\item les bases de données SpatiaLite (lire la section \ref{label_spatialite}),
\item les formats raster et vecteur provenant des bases de données GRASS, 
\item les données spatiales provenant des services réseaux compatibles OGC 
comme le Web Map Service (WMS) ou le Web Feature Service (WFS) (voir la section 
\ref{working_with_ogc}),
\item les données OpenStreetMap (voir la section \ref{plugins_osm}).
\end{itemize}

\minisec{Parcourir les données et créer des cartes} 

Vous pouvez créer des cartes et les parcourir de manière interactive avec une 
interface intuitive. Les outils disponibles dans l'interface sont :

\begin{itemize}[label=--]
\item projection à la volée (adapte les unités de mesure et reprojette 
automatiquement les données vectorielles)
\item composition de carte
\item panneau de navigation
\item signet géospatial
\item identification et sélection des entités
\item affichage, édition et recherche des attributs
\item étiquetage des entités
\item personnalisation de la sémiologie des données raster et vecteur
\item ajout d'une couche de graticule lors de la composition
\item ajout d'une barre d'échelle, d'une flèche indiquant le nord et d'une étiquette de droits d'auteur
\item sauvegarde et chargement de projets
\end{itemize}

\minisec{Créer, éditer, gérer et exporter des données}

Vous pouvez créer, éditer, gérer et exporter des données vectorielles dans 
plusieurs formats. \qg permet ce qui suit :

\begin{itemize}[label=--]
\item numérisation pour les formats gérés par OGR et les couches vectorielles de 
GRASS
\item création et édition des fichiers de forme (shapefiles), des couches 
vectorielles de GRASS et des tables géométriques SpatiaLite.
\item géoréférencement des images avec l'extension de géoréférencement
\item importation, exportation du format GPX pour les données GPS, avec la 
conversion des autres formats GPS vers le GPX ou l'envoi, la réception 
directement vers une unité GPS (le port USB a été ajouté à la liste des ports 
utilisables sous \nix{})
\item visualisation et édition des données OpenStreetMap
\item création de couches \pg à partir de fichiers shapefiles grâce au module 
d'extension SPIT
\item prise en charge améliorée des tables PostGIS
\item gestion des attributs des couches vectorielles grâce à l'extension de 
gestion des tables ou celle de tables attributaires (voir la section 
\ref{sec:attribute table})
\item enregistrer des captures d'écran en tant qu'images géoréférencées
\end{itemize}

Les couches raster doivent être importées dans GRASS pour pouvoir être éditées 
et exportées vers d'autres formats.

\minisec{Analyser les données} 

Vous pouvez opérer des analyses spatiales sur des données \ppg et autres formats 
OGR en utilisant l'extension fTools. \qg permet actuellement l'analyse 
vectorielle, l'échantillonnage, la gestion de la géométrie et des bases de 
données. Vous pouvez aussi utiliser les outils GRASS intégrés qui comportent 
plus de 400 modules (voir la section \ref{sec:grass})

\minisec{Publier une carte sur Internet}

\qg peut être employé pour exporter des données vers un mapfile et le publier 
sur Internet via un serveur web employant l'UMN MapServer. \qg peut aussi 
servir de client WMS/WFS ou de serveur WMS.

\minisec{Étendre les fonctionnalités de \qg grâce à des extensions} 

\qg peut être adapté à vos besoins particuliers du fait de son architecture 
extensible à base de modules. \qg fournit des bibliothèques qui peuvent être 
employées pour créer des extensions, vous pouvez même créer de nouvelles 
applications en C++ ou python !

\minisec{Extensions principales}

\begin{enumerate}
\item Ajouter une couche de texte délimité (charge et affiche des fichiers 
texte ayant des colonnes contenant des coordonnées X/Y)
\item Capture de coordonnées (Enregistre les coordonnées sous la souris dans 
un SCR différent)
\item Décorations (Étiquette de droit d'auteur, flèche indiquant le nord et 
barre d'échelle)
\item Insertion de diagrammes (place des diagrammes sur une couche vectorielle)
\item Extension déplacement (gère le déplacement de point lorsqu'ils ont la même 
position)
\item Convertisseur Dxf2Shp (convertit les fichiers DXF en fichier SHP)
\item Outils GPS (importe et exporte des données GPS)
\item GRASS (intégration du SIG GRASS)
\item GDALTools (Outils GDAL intégrés dans QGIS)
\item Géoréférenceur GDAL (ajoute une projection à un raster)
\item Extension d'interpolation (interpole une surface en utilisant les sommets 
d'une couche vectorielle)
\item Charger des couches raster de PostGIS vers QGIS 
\item Export Mapserver (exporte un fichier de projet QGIS dans le format de 
carte de MapServer)
\item Édition Offline (Permet l'édition offline et la synchronisation avec une 
base de données)
\item Extension OpenStreetMap (permet de visualiser et d'éditer des données OSM)
\item support des GeoRaster d'Oracle Spatial
\item Installateur d'extensions (télécharge et installe des extensions 
python pour \qg)
\item Analyse de terrain raster
\item Extension graphe routier (analyse réseau du plus court chemin)
\item SPIT, outil d'importation de Shapefile vers \ppg
\item Extension SQL Anywhere (Store des couches vectorielles dans une base SQL 
Anywhere)
\item Extension Requête spatiale (réalise des requêtes spatiales sur des couches 
vectorielles)
\item Ajouter une couche WFS 
\item eVIS (visualisation d'évènements multimédias)
\item fTools (outils d'analyse et de gestion de vecteurs)
\item Console Python (accédant à l'environnement QGIS)
\end{enumerate}

%\minisec{External Python Plugins}
\minisec{Extensions Python externes}

\qg offre un nombre croissant d'extensions complémentaires en Python fournies 
par la communauté. Ces extensions sont entreposées dans le répertoire 
UTILISATEUR\footnote{L'emplacement change selon le système d'exploitation, 
ainsi sous \nix{} il s'agit du répertoire HOME tandis que sous \win{} il 
s'agit du répertoire utilisateur se situant dans Document And Settings}/.qgis/python/plugins 
et peuvent être facilement installées en utilisant l'extension d'installation 
Python (voir la section \ref{sec:plugins}). 

\subsubsection{Quoi de neuf dans la version ~\CURRENT} 

Veuillez noter que cette version est un jalon important dans la série des 
publications. Comme tel, elle incorpore de nouvelles fonctionnalités et étend 
l'interface de programmation par rapport à \qg 1.0.x et \qg 1.6.0. Nous 
recommandons d'utiliser cette version préférentiellement aux précédentes.

Cette version n'inclut pas moins de 177 résolutions de problèmes, ainsi que des 
améliorations et de nouvelles fonctionnalités.

\minisec{Style, étiquetage et diagrammes}

\begin{itemize}[label=--]
%\item New symbology now used by default!
\item Nouveau système de style utilisé maintenant par défaut !
%\item Diagram system that uses the same smart placement system as labeling-ng
\item Système de diagramme qui utilise le même système de placement intelligent 
que l'étiquetage-ng 
%\item Export and import of styles (symbology-ng).
\item Exportation et importation de styles (symbologie-ng). 
%\item Labels for rules in rule-based renderers.
\item Étiquettes pour les règles pour le rendu basé sur des règles. 
%\item Font marker can have an X,Y offset.
\item Le marqueur de police peut avoir un décalage X, Y. 
%\item Line symbology:
\item Style de ligne :
\begin{itemize}[label=--]
%\item Option to put marker on the central point of a line.
\item Option pour mettre des marqueurs sur le point central d'une ligne. 
%\item Option to put marker only on first/last vertex of a line.
\item Option pour mettre un marqueur seulement sur le premier/dernier sommet 
d'une ligne. 
%\item Allow the marker line symbol layer to draw markers on each vertex.
\item Autoriser la couche de symbole de marqueur linéaire à dessiner des 
marqueurs sur chaque sommet. 
\end{itemize}
%\item Polygon symbology:
\item Style de polygone :
\begin{itemize}[label=--]
%\item Rotation for svg fills.
\item Rotation pour le remplissage svg. 
%\item Added 'centroid fill' symbol layer which draws a marker on polygon's centroid.
\item Ajout de la couche de symbole 'remplissage de centroïde' qui dessine un 
marqueur sur le centroïde d'un polygone. 
%\item Allow the line symbol layers to be used for outline of polygon (fill) symbols.
\item Permet l'utilisation des couches de symbole linéaire pour les symboles de 
contours de polygones (remplissages). 
\end{itemize}
%\item Labels
\item Étiquetage
\begin{itemize}[label=--]
%\item Ability to set label distance in map units.
\item Possibilité d'ajuster la longueur de l'étiquette dans les unités de la carte.
%\item Move/rotate/change label edit tools to interactively change data defined label properties.
\item Déplacement/rotation/changement des outils d'édition d'étiquette pour modifier 
interactivement les propriétés des étiquettes de données définies.
\end{itemize}
%\item New Tools
\item Nouveaux outils
\begin{itemize}[label=--]
%\item Added GUI for gdaldem.
\item Ajout d'une interface graphique utilisateur (GUI) pour gdaldem. 
%\item Added field calculator with functions like \$x, \$y and \$perimeter.
\item Ajout de la calculatrice de champ avec des fonctions comme \$x, \$y et \$périmètre. 
%\item Added 'Lines to polygons' tool to vector menu.
\item Ajout de l'outil 'Lignes vers polygones" au menu Vecteur. 
%\item Added voronoi polygon tool to Vector menu.
\item Ajout de l'outil "Polygone de Voronoi" au menu Vecteur. 
\end{itemize}
\end{itemize}

%\minisec{User interface updates}
\minisec{Mises à jour de l'interface utilisateur}

\begin{itemize}[label=--]
%\item Allow managing missing layers in a list.
\item Permettre la gestion des couches manquantes dans une liste. 
%\item Zoom to group of layers.
\item Zoom sur un groupe de couches. 
%\item 'Tip of the day' on startup. You can en/disable tips in the options panel.
\item "Astuce du jour" au démarrage. Vous pouvez activer/désactiver les astuces 
dans le panneau des options. 
%\item Better organisation of menus, separate database menu added.
\item Meilleure organisation des menus, ajout d'un menu séparé "Base de données". 
%\item Add ability to show number of features in legend classes. Accessible via right-click legend menu.
\item Ajout de la possibilité d'afficher le nombre d'entités dans les classes de 
légende. Accessible via le menu de légende par le clic droit.
%\item General clean-ups and usability improvements.
\item Nettoyage général et améliorations d'utilisation.
\end{itemize}

%\minisec{CRS Handling}
\minisec{Gestion de la projection}

\begin{itemize}[label=--]
%\item Show active crs in status bar.
\item Voir la projection active dans la barre d'état. 
%\item Assign layer CRS to project (in the legend context menu).
\item Attribuer la projection d'une couche dans le projet (dans le menu contextuel 
de la légende). 
%\item Select default CRS for new projects.
\item Sélectionner une projection par défaut pour les nouveaux projets. 
%\item Allow setting CRS for multiple layers at once.
\item Permettre le paramétrage d'une projection pour plusieurs couches à la fois. 
%\item Default to last selection when prompting for CRS.
\item Par défaut dernière sélection lors de la demande de la projection. 
\end{itemize}

%\minisec{Rasters}
\minisec{Rasters}

\begin{itemize}[label=--]
%\item Added AND and OR operator for raster calculator
\item Ajout de l'opérateur ET et OU pour la calculatrice raster.
%\item '''On-the-fly reprojection of rasters added!'''
\item \textbf{Ajout de la reprojection à la volée des rasters !} 
%item Proper implementation of raster providers.
\item Implémentation correcte des fournisseurs de raster. 
%\item Added raster toolbar with histogram stretch functions.
\item Ajout de la barre d'outils avec des fonctions raster d'étirement de l'histogramme. 
\end{itemize}

\minisec{Pilotes et Prestataires de données}

\begin{itemize}[label=--]
%\item New SQLAnywhere vector provider.
\item Nouveau connecteur de données vecteur SQLAnywhere.
%\item Table join support
\item Support de la jointure de tables
%\item Feature form updates
\item Mises à jour du formulaire d'entités
%\item Make NULL value string representation configurable.
\item Permettre la configuration la représentation de la chaîne de valeur NULL
%\item Fix feature updates in feature form from attribute table.
\item Correction de la mise à jour d'entités sous forme d'entités de la table 
d'attributs
%\item Add support for NULL values in value maps (comboboxes).
\item Ajout de la gestion pour les valeurs NULL dans les cartes de valeur (comboboxes)
%\item Use layer names instead of ids in drop down list when loading value maps 
%from layers.
\item Utilisation des noms de couches au lieu des identifiants dans la liste 
déroulante lors du chargement des cartes de valeur à partir de couches
%\item Support feature form expression fields: line edits on the form which name 
%prefix 'expr\_' are evaluated. Their value is interpreted as field calculator 
%string and replaced with the calculated value.
\item Gestion des champs d'expression à partir du formulaire d'entités : les éditions 
des lignes dans le formulaire dont le préfixe 'expr\_' sont évaluées. Leur valeur 
est interprétée comme une chaîne issue du calculateur de champs et remplacée par 
la valeur calculée. 
%\item Support searching for NULL in attribute table.
\item Gestion de la recherche NULL dans la table d'attributs 
%\item Attribute editing improvements
\item Améliorations de l'édition des attributs 
%\item Improved interactive attribute editing in table (adding/deleting features, 
%attribute update).
\item Amélioration de l'édition interactive d'attributs dans la table 
(ajout/suppression d'entités, mise à jour d'attribut). 
%\item Allow adding of geometryless features.
\item Permettre l'ajout d'entités sans géométrie 
%\item Fixed attribute undo/redo.
\item Correction de la fonction undo/redo sur un attribut. 
%\item Improved attribute handling.
\item Amélioration de la manipulation d'attributs. 
%\item Optionally re-use entered attribute values for next digitized feature.
\item Eventuellement réutilisation des valeurs d'attribut renseignées pour les 
entités suivantes numérisées. 
%\item Allow merging/assigning attribute values to a set of features.
\item Autorisation de la fusion/assignation des valeurs d'attribut à un ensemble 
d'entités. 
%\item Allow OGR 'save as' without attributes (for eg. DGN/DXF).
\item Permettre à OGR d'"enregistrer sous" sans attributs (par exemple pour DGN/DXF). 
\end{itemize}

%\minisec{Api and Developer Centric}
\minisec{API et outils pour développeurs}

\begin{itemize}[label=--]
%\item Refactored attribute dialog calls to QgsFeatureAttribute.
\item  Réécriture des appels à la boîte de dialogue d'attributs pour QgsFeatureAttribute. 
%\item Added QgsVectorLayer::featureAdded signal.
\item Ajout du signalement QgsVectorLayer::featureAdded. 
%\item Layer menu function added.
\item Ajout de la fonction de menu Couche. 
%\item Added option to load c++ plugins from user specified directories. 
%Requires application restart to activate.
\item Ajout d'une option pour charger des extensions C++ à partir de répertoires 
utilisateur définie. Nécessite le redémarrage de l'application pour l'activer. 
%\item Completely new geometry checking tool for fTools. Significantly faster, 
%more relevant error messages, and now supports zooming to errors. See the new 
%QgsGeometry.validateGeometry function
\item Outil entièrement nouveau de vérification de la géométrie pour fTools. 
Beaucoup plus rapide, messages d'erreur plus pertinents, et gère désormais le 
zoom sur les erreurs. Voir la nouvelle fonction QgsGeometry.validateGeometry.
\end{itemize}

%\minisec{QGIS Server}
\minisec{QGIS Server}

\begin{itemize}[label=--]
%\item Ability to specify wms service capabilities in the properties section of the project file (instead of wms\_metadata.xml file).
\item Possibilité de spécifier des capacités de service WMS dans la section des propriétés du fichier de projet (au lieu du fichier wms\_metadata.xml). 
%\item Support for wms printing with GetPrint-Request.
\item Gestion de l'impression wms avec GetPrint-Request. 
\end{itemize}

%\minisec{Plugins}
\minisec{Extensions}

\begin{itemize}[label=--]
%\item Support for icons of plugins in the plugin manager dialog.
\item Gestion des icônes des extensions dans la boîte de dialogue de l'installeur 
d'extensions. 
%\item Removed quickprint plugin - use easyprint plugin rather from plugin repo.
\item Suppression de l'extension QuickPrint – Utilisez plutôt l'extension Easyprint dans l'entrepôt des extensions.
%\item Removed ogr convertor plugin - use 'save as' context menu rather.
\item Suppression de l'extension ogr convertor – Utilisez plutôt le menu 
contextuel 'Enregistrer sous'.
\end{itemize}

%\minisec{Printing}
\minisec{Impression}

\begin{itemize}[label=--]
%\item Undo/Redo support for the print composer
\item Gestion de l'annulation/refaire pour le composeur d'impression
\end{itemize}

\newpage


