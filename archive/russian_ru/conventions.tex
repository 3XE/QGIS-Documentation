%  !TeX  root  =  user_guide.tex

% when the revision of a section has been finalized,
% comment out the following line:
% \updatedisclaimer

\addchap{Элементы}\label{label_conventions}

В этом разделе описывается набор стандартных стилистических элементов,
принятых в документе. В данном руководстве пользователя
используются следующие элементы:

\addsec{Элементы интерфейса пользователя}

Элементы интерфейса пользователя используются для имитации внешнего
вида интерфейса пользователя. Задача элементов"--- дать наглядное
представление, так, чтобы пользователь мог посмотреть на интерфейс и
найти то, что описано в инструкции руководства.

\begin{itemize}[label=--,itemsep=5pt]
\item Пункты меню: \mainmenuopt{Слой} \arrow
\dropmenuopttwo{mActionAddRasterLayer}{Добавить растровый слой}

или

\mainmenuopt{Вид} \arrow
\dropmenuopt{Панели инструментов} \arrow \dropmenucheck{Оцифровка}
\item Инструмент: \toolbtntwo{mActionAddRasterLayer}{Добавить растровый слой}
\item Кнопка: \button{По умолчанию}
\item Заголовок диалогового окна: \dialog{Свойства слоя}
\item Вкладка: \tab{Общие}
\item Набор инструментов: \toolboxtwo{nviz}{nviz"--- 3D-визуализация}
\item Флажок: \checkbox{Отрисовка}
\item Переключатель: \radiobuttonon{Postgis SRID} \radiobuttonoff{EPSG ID}
\item Выбрать число: \selectnumber{{}Тон}{60}
\item Выбрать строку: \selectstring{{}Стиль обводки}{---Сплошная}
\item Выбрать файл: \browsebutton
\item Выбрать цвет: \selectcolor{{}Цвет обводки}{yellow}
\item Ползунок: \slider{Прозрачность}
\item Ввод текста: \inputtext{{}Имя в легенде}{lakes.shp}
\end{itemize}
Затенение указывает на интерактивный компонент графического интерфейса.

\addsec{Текстовые элементы или клавиатурные сокращения}

Руководство также включает в себя стили, связанные с текстом,
клавиатурными сокращениями и примерами кода для обозначения различных
сущностей, таких, как классы или методы. Они не обязательно соответствуют
каким-либо элементам интерфейса.

\begin{itemize}[label=--]
%Use for all urls. Otherwise, it is not clickable in the document.
\item Гиперссылки: \url{http://qgis.org}
%\item Single Keystroke: press \keystroke{p}
\item Комбинации клавиш: нажать \keystroke{Ctrl+B} означает
нажать и удерживать клавишу Ctrl, а затем нажать клавишу B.
\item Название файла: \filename{lakes.shp}
%\item Name of a Field: \fieldname{NAMES}
\item Название класса: \classname{NewLayer}
\item Метод: \method{classFactory}
\item Имя сервера: \server{myhost.de}
%\item SQL Table: \sqltable{example needed here}
\item Текст, вводимый пользователем: \usertext{qgis ---help}
\end{itemize}

Примеры кода отображаются с помощью шрифта фиксированной ширины:
\begin{verbatim}
PROJCS["NAD_1927_Albers",
  GEOGCS["GCS_North_American_1927",
\end{verbatim}

\addsec{Инструкции, специфичные для конкретных платформ}

Последовательности команд интерфейса пользователя и краткие описания
 могут быть представлены в виде строки: Нажмите \{\nix{}\win{Файл}
\osx{QGIS}\} \arrow Выход, чтобы закрыть QGIS.

Это означает, что на платформах Linux, Unix и Windows сначала нужно
выбрать пункт меню <<Файл>>, а затем в выпадающем меню щелкнуть <<Выход>>,
в то время как в Mac~OSX сначала нужно выбрать меню \qg,
а затем в выпадающем меню выбрать Выход. Если нужно большее количество
текста, оно может быть представлено списком:

\begin{itemize}
\item \nix{сделать это;}
\item \win{сделать то;}
\item \osx{сделать что-то еще.}
\end{itemize}

или в виде абзацев.

\nix{} \osx{} Сделать это, и это, и это. И так далее, и тому подобное\ldots

\win{}Сделать то. И еще то и то. И так далее, и тому подобное\ldots

Снимки экрана, которые встречаются в руководстве пользователя, были
созданы на разных платформах; платформа обозначается специальной иконкой
в конце подписи к рисунку.

Русскоязычное руководство использует снимки экрана, выполненные в
операционной системе Windows.
