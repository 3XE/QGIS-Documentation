%  !TeX  root  =  user_guide.tex

\chapter{Справка и поддержка}\label{label_helpsupport}

% when the revision of a section has been finalized,
% comment out the following line:
% \updatedisclaimer

\section{Списки рассылки}
QGIS находится в состоянии активной разработки и поэтому иногда может
работать не так, как вы ожидаете. Подписка на рассылку qgis-users
является наиболее предпочтительным способом получения помощи. Ваш вопрос
будет доступен широкой аудитории, а ответы смогут помочь другим.

\minisec{qgis-users}
Список рассылки предназначен как для обсуждения QGIS в целом, так и для
специфических вопросов, касающихся установки и использования.
Подписаться на список рассылки qgis-users можно посетив следующий URL: \\
\url{http://lists.osgeo.org/mailman/listinfo/qgis-user}

\minisec{fossgis-talk-liste}
Для говорящих на немецком немецкая группа FOSSGIS e.V. организовала
рассылку fossgis-talk-liste. Этот список рассылки предназначен для
обсуждения свободных ГИС в целом, включая QGIS. Вы можете подписаться на
список рассылки fossgis-talk-liste, посетив URL: \\
\url{https://lists.fossgis.de/mailman/listinfo/fossgis-talk-liste}

\minisec{qgis-developer}
Если вы разработчик и сталкиваетесь с проблемами более технического
характера, то, возможно, захотите присоединиться к рассылке
qgis-developer здесь:\\
\url{http://lists.osgeo.org/mailman/listinfo/qgis-developer}

\minisec{qgis-commit}
Каждый раз, когда выполняется изменение в коде QGIS, в этот список
рассылки отправляется сообщение. Если вы хотите быть в курсе всех
изменений в коде, подпишитесь на эту рассылку:\\
\url{http://lists.osgeo.org/mailman/listinfo/qgis-commit}

\minisec{qgis-trac}
Эта рассылка оповещает о событиях, связанных с управлением проектом, в
том числе, сообщениях об ошибках, задачах и пожеланиях. Подписаться на
рассылку можно по адресу:\\
\url{http://lists.osgeo.org/mailman/listinfo/qgis-trac}

\minisec{qgis-community-team}
Этот список рассылки посвящён таким вопросам, как документация,
контекстная справка, руководство пользователя, онлайн ресурсы (веб-сайт,
блог, списки рассылки, форумы) и перевод. Если вы хотите поработать над
руководством пользователя, то этот список рассылки является тем местом,
где нужно задавать свои вопросы. Подписаться на этот список: \\
\url{http://lists.osgeo.org/mailman/listinfo/qgis-community-team}

\minisec{qgis-release-team}
Рассылка служит для обсуждения вопросов, связанных с выпуском новых
версий, подготовкой бинарных пакетов для различных ОС и для
анонсирования новых выпусков. Чтобы подписаться на рассылку посетите
следующий адрес:\\
\url{http://lists.osgeo.org/mailman/listinfo/qgis-release-team}

\minisec{qgis-tr}
Список рассылки посвящённый вопросам перевода. Если вы хотите работать над
переводом руководств или интерфейса пользователя (GUI), то все свои вопросы
нужно задавать здесь. Подписаться на рассылку можно по адресу:\\
\url{http://lists.osgeo.org/mailman/listinfo/qgis-tr}

\minisec{qgis-edu}
Этот список рассылки обсуждаются вопросы обучения работе с QGIS. Если вы
желаете заняться разработкой обучающих материалов, то эта рассылка будет
хорошей отправной точкой. Чтобы подписаться на рассылку посетите следующий
адрес:\\
\url{http://lists.osgeo.org/mailman/listinfo/qgis-edu}

\minisec{qgis-psc}
Список рассылки используется Руководящим комитетом для обсуждения
вопросов, связанных с общим управлением и направлением развития Quantum
GIS. Подписаться на рассылку можно здесь:\\
\url{http://lists.osgeo.org/mailman/listinfo/qgis-psc}

Вы можете подписаться на любой из вышеуказанных списков. Пожалуйста, не
забывайте участвовать в рассылках, отвечая на вопросы и делясь опытом.
Также обратите внимание, что рассылки qgis-commit и qgis-trac служат
только для оповещения и не предназначены для писем пользователей.

\section{IRC}
Нас можно найти в IRC~--- посетите наш канал \#qgis на
\url{irc.freenode.net}. Пожалуйста, задав вопрос, немного подождите,
посетители канала могут быть заняты другими делами, и им потребуется
некоторое время, чтобы увидеть ваш вопрос. Кроме того, доступна
коммерческая поддержка QGIS. Больше информации вы найдете на нашем
веб-сайте \url{http://qgis.org/en/commercial-support.html}.

Если вы пропустили обсуждение в IRC, это не проблема! Мы записываем все
обсуждения, поэтому вы всегда можете наверстать упущенное. Просто
перейдите по ссылке \url{http://logs.qgis.org} и прочитайте журналы IRC.

\section{Багтрекер}
Так как список рассылки qgis-users полезен для общих вопросов типа <<как я
могу сделать xyz в QGIS>>, вам может потребоваться сообщить нам об
ошибках в QGIS. Сделать это можно, используя баг-трекер QGIS
\url{http://hub.qgis.org/projects/quantum-gis/issues}. Пожалуйста, при создании нового
сообщения об ошибке, оставляйте адрес электронной почты, используя
который, мы сможем обратиться к вам за дополнительной информацией.

Имейте в виду, что ваша ошибка не всегда будет имет приоритет, который
бы вам хотелось (в зависимости от сложности). Исправление некоторых
ошибок может потребовать значительных усилий от разработчика и большого
количества времени, а всё это не всегда есть в наличии.

Предложения по усовершенствованию можно отправлять, используя ту же
систему, что и ошибки. Пожалуйста, убедитесь, что для сообщения указан
тип \usertext{enhancement}.

Если вы нашли ошибку и исправили ее самостоятельно, можете отправить
этот патч нам. Для этого снова воспользуйтесь системой trac
\url{http://hub.qgis.org/projects/quantum-gis/issues}. Выберите тип \usertext{patch} для
сообщения. Кто-нибудь из разработчиков рассмотрит его и применит. \\
Пожалуйста, не волнуйтесь, если ваш патч не был применен сразу~---
разработчики могут быть заняты другим.

% unused, since community.qgis.org seems to be lost. (SH)
% There is also a community site for QGIS where we encourage QGIS users to share
% their experiences and provide case studies about how they are using QGIS. The
% community site is available at: http://community.qgis.org

\section{Блог}
Сообщество QGIS также ведет блог \url{http://www.qgis.org/planet},
где вы можете найти статьи, интересные как пользователям, так и разработчикам,
а также материалы из других блогов сообщества. Мы приглашаем вас принять
участие и добавить свой блог о QGIS в ленту новостей!

\section{Wiki}
И наконец, мы поддерживаем Wiki \url{http://www.qgis.org/wiki}, где
можно найти множество полезной информации, касающейся разработки QGIS,
планы по выпуску, ссылки на загрузку, советы по переводу и т.\,д.
Проверьте сами, и найдете много интересного!
