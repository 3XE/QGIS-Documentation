%  !TeX  root  =  user_guide.tex
\pagestyle{scrheadings}
\chapter{GIS概述}\label{label_intro}

% when the revision of a section has been finalized, 
% comment out the following line:
%\updatedisclaimer

地理信息系统(GIS)(\cite{mitchel05}\footnote{本章由Tyler
Mitchell创作(\url{http://www.oreillynet.com/pub/wlg/7053})并可以在通用创新许可(Creative Common Licenses)下使用。Tyler编著过 \textit{Web Mapping Illustrated}, O'Reilly出版社出版, 2005.})是一个软件的集合,它允许用户创建,显示,查询及分析地理空间数据。地理空间数据是指关于一个实体的地理位置的信息。这往往涉及到地理坐标的使用,如经度、纬度值。空间数据是另一种常用术语,如地理数据、GIS数据、地图数据、位置数据、坐标数据和空间几何数据。

地理空间数据的应用程序可执行很多功能。地图制作是最容易理解的地理空间应用程序的功能。制图程序接受地理空间数据并且以可视化的形式将其渲染,通常是在计算机屏幕或者是打印页面上。应用程序可以呈现静态的地图(一个简单的图像)或者动态的地图(通过查看一个桌面程序或者网页地图制作而成)。

很多人误认为地理空间应用程序仅仅是制作地图,但是其另一个主要功能是空间数据分析。一些典型的分析类型包括:

\begin{enumerate}
\item 地理位置间的距离计算
\item 特定地理区域内的面积量算
\item 地理图元的叠置状态识别
\item 地理图元间的叠置分析
\item 某地理实体周围一定范围内的实体数量计算
\item 其他。。。
\end{enumerate}

这些看似简单,但是可以通过不同的方法在众多学科领域得到应用。分析的结果可显示在地图上,但是在支持管理决策的报告中往往以表格的形式呈现。
        
移动定位服务开始承诺引进其他各种服务,但是大部分将基于地图和分析相结合。例如,您有一部手机可以追踪您的地理位置。如果您有适当的软件,您的手机就可以告诉您在步行距离内有什么样的餐厅。虽然这是一项新的地理空间技术应用,但其本质是进行地理空间数据分析并为用户列出结果。

\section{为何出现这么多新特性?}\label{label_whynew}

其实事实并非如此。有许多新的硬件设备使移动地理信息服务成为可能。许多开源的地理空间应用程序也是可以使用的,但是已存在的集中的地理空间硬件和软件并没有什么更新之处。全球定位系统(GPS)接收器正变得司空见惯,但是已经被应用于各个行业超过十年。同样的,桌面制图和分析工具也已经成为一个主要的商业市场,主要集中在工业领域比如自然资源管理。
        
其更新之处在于最新的硬件和软件怎样被应用和由谁应用。制图和分析工具的传统使用者是训练有素的GIS分析员或者是使用类似CAD工具进行数学绘图的技术人员。现如今,家用电脑和开源软件(OSS)包的处理能力已经可以使业余爱好者,专业人士以及web开发人员等成为一个与地理空间数据进行交互的团队。学习曲线有所回落,成本下降,地理空间技术的饱和量有所增加。
        
如何保存地理空间数据?简单的说,现今在地理空间应用程序中除了传统的列表数据外还有有两种数据格式被广泛使用。

\subsection{栅格数据}\label{label_rasterdata}

地理空间数据格式类型之一是栅格数据或者简称“栅格”。最易识别的栅格数据格式是数字卫星影像和航空照片。高程阴影或者高程数字模型也是典型的栅格数据。任何类型的地图特征都可以用栅格数据表示,但是有局限性。
        
栅格数据是由单元格构成的规则网格或者由像素构成的影像,具有固定的行和列。每个单元格都有一个数值和确定的地理面积(例如30*30米)。

多个重叠的栅格图像是用来表示超过一个颜色值的图像(也就是说,不同红色、绿色和蓝色值的栅格图层合成之后可以生成一幅彩色图像)。卫星影像也是由多重“波段”合成的数据。每个波段基本上是一个独立的空间重叠的栅格,其中每个波段都拥有特定的光波长。正如你能想像到的,一个大型栅格数据占用更多的文件空间。一个具有较小单元格的栅格数据可以提供更多的细节,但是需要更多的文件空间。诀窍在于找到一个适当的平衡使单元格的大小既易存储又能保证分析和制图。

\subsection{矢量数据}\label{label_vectordata}

矢量数据也被应用于地理空间的应用程序。如果你能理解三角学和坐标几何类就已经了解了一些矢量数据的特性。简单的理解就是矢量是通过坐标集合来描述位置的一种方式。每一个坐标是指用X-Y坐标系表示的地理位置。
        
这可以被认为是一个直角平面——即你在学校中学到的用X轴和Y轴显示的图表。你可能已经用它们绘制过退休储蓄下降表或者复合抵押贷款利息上升表,但其概念是地理空间数据分析和制图的基础。
        
目的不同刚表示地理坐标的方法就不同。地图投影是一个需要专门去学习的领域。

矢量数据有三种表现形式,每一种都是建立在前一种的基础之上并且更加复杂。

\begin{enumerate}
\item 点 —— 一个代表离散的地理位置的坐标对(X,Y)
\item 线 —— 由特定顺序的多个坐标对((X1,Y1),(X2,Y2),(X3,Y3)。。。(Xn,Yn))组成的坐标串,正如一直从(X1,Y1)、(X2,Y2)到(Xn,Yn)画一条直线。各点间的部分称为线段。线段可以有长度,并且依赖于各点的顺序可以定义其方向。 严格来讲,线是指两个点相连形成的部分,而折线是联系在一起的多条线的合称。  
\item 多边形 —— 当线由多余两个的点组成,而且最后一个点与第一个点重合时,我们将对象称为多边形。三角形、圆形、矩形等都是多边形。多边形的关键在于其有固定的面积。
\end{enumerate}
