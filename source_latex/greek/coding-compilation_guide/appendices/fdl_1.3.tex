\section{\rlap{Ελεύθερη Άδεια Τεκμηρίωσης GNU}}\label{label_fdl}
\index{license!FDL}
 \begin{center}

       Έκδοση 1.3, 3 Νοέμβριος 2008


 Copyright \copyright{} 2000, 2001, 2002, 2007, 2008  Free Software Foundation, Inc.
 
 \bigskip
 
     <http://fsf.org/>
  
 \bigskip
 
 Οποιοδήποτε έχει την άδεια να αντιγράψει και να διανέμει αυτολεξεί αντίγραφα αυτής της άδειας εγγράφου, αλλά η αλλαγή του δεν επιτρέπεται.
\end{center}


\begin{center}
{\bf\large Εισαγωγή}
\end{center}

Ο σκοπός αυτής της Άδειας είναι να δημιουργηθεί ένα εγχειρίδιο, βιβλίο κειμένου, ή άλλο λειτουργικό και χρήσιμο έγγραφο ελεύθερο με την έννοια της “ελευθερίας”: για να βεβαιώσουμε τους πάντες για την αποτελεσματική ελευθερία να το αντιγράψετε και να το αναδιανείμετε, τροποποιώντας το ή όχι, είτε εμπορικά έιτε όχι, Επιπλέον, αυτή η Άδεια διατηρεί για τον συγγραφέα και τον εκδότη έναν τρόπο για να ανταμειφθούν για τη δουλειά τους, και ταυτόχρονα να μη θεωρούνται υπεύθυνοι για τροποιήσεις που έγιναν απο άλλους.

Αυτή η Άδεια είναι ένα “copyleft”, που σημαίνει ότι παράγωγες δουλειές του εγγράφου πρέπει να είναι ελεύθερες με την ίδια έννοια. Συμπληρώνει την GNU GPL άδεια, η οποία είναι ένα μια copyleft άδεια σχεδιασμένη για ελεύθερο λογισμικό.

Έχουμε σχεδιάσει αυτή την άδεια έτσι ώστε να τη χρησιμοποιήσουμε για εγχειρίδια ελεύθερου λογισμικού, γιατι το ελεύθερο λογισμικό χρειάζεται ελεύθερη τεκμηρίωση: ένα ελεύθερο πρόγραμμα θα έπρεπε να έρχεται μαζί με εγχειρίδια παρέχοντας τις ίδιες ελευθερίες που το λογισμικό κάνει. Αλλά αυτή η Άδεια δεν περιορίζεται σε εγχειρίδια λογισμικού; μπορεί να χρησιοποιηθεί για οποιαδήποτε εργασία αναφερεται σε κείμενο, ανεξάρτητα απο το θέμα ή απο το αν έχει εκδοθεί σαν εκτυπωμένο βιβλίο. Προτείνουμε αυτή την Άδεια πρωταρχικά για εργασίες των οποίων ο σκοπός είναι οι οδηγίες ή η βοήθεια.


\begin{center}
{\Large\bf 1. ΕΦΑΡΜΟΣΙΜΟΤΗΤΑ ΚΑΙ ΟΡΙΣΜΟΙ\par}
\end{center}

Αυτή η Άδεια ισχύει για κάθε εγχειρίδιο ή άλλη εργασία, σε κάθε μέσο, που περιέχει μια σημείωση τοποθετημένη απο τον ιδιοκτήτη των πνευματικών δικαιωμάτων που θα λέει ότι μπορεί να διανεμηθεί υπο τους όρους αυτής της Άδειας. Μια τέτοια σημείωση παρέχει μια παγκόσμια, ελεύθερη απο πνευματικά διακαιώματα άδεια, απεριόριστη σε διάρκεια, για χρήση αυτής της εργασίας υπο τους όρους που δηλώνονται εδώ. Το ``\textbf{Έγγραφο}'', παρακάτω, αναφέρεται σε οποιοδήποτε τέτοιο εγχειρίδιο. Οποιοδήποτε μέλος του κοινού είναι κάτοχος άδειας, και αναφέρεται ως ``\textbf{εσείς}''.  Αποδέχεστε την άδεια αν αντιγράψετε, τροποποήσετε ή αναδιανέμετε την εργασία με έναν τρόπο που απαιτεί άδεια υπο το νόμο των πνευματικών δικαιωμάτων.

Μια ``\textbf{Τροποποιημένη Έκδοση}'' του εγγράφου σημαίνει οποιαδήποτε εργασία εμπεριέχει το Έγγραφο ή μέρος της, είτε αντεγραμμένα λεκτικά, ή με τροποποιήσεις και ή μετεφρασμένη σε κάποια άλλη γλώσσα.

Ένα ``\textbf{Δευτερεύων Κεφάλαιο}'' είναι ένα ονομασμένο παράρτημα ή το εισαγωγικό κεφάλαιο του Εγγράφου που ασχολείται αποκλειστικά με τη σχέση των εκδοτών ή των συγγραφέων του Εγγράφου με το συνολικό θέμα του Εγγράφου (ή με σχετικά θέματα) και δεν εμπεριέχει τίποτα που θα μπορούσε να σχετιστεί άμεσα με το συνολικό θέμα (Οπότε, αν το έγγραφο είναι εν μέρη ένα βιβλίο μαθηματικών, ένα Δευτερεύων Κεφάλαιο μπορεί να μην εξηγεί μαθηματικά). Η σχέση θα μπορούσε να είναι ιστορικής σύνδεσης με το θέμα ή με σχετικά θέματα, ή νομικού, εμπορικού, φιλοσοφικού, συναφούς ή πολιτικής θέσης που να σχετίζεται μαζί τους.

Τα ``\textbf{Παρεμφερή Κεφάλαια}'' είναι σαφώς δευτερεύοντα κεφάλαια των οποίων οι τίτλοι έχουν καθοριστεί, ως των Παρεμφερών Κεφαλαίων, στη σημείωση που λέει ότι το έγγραφο έχει εκδοθεί υπο αυτή την Άδεια. Αν ένα κεφάλιο δεν ταιριάζει με τον παραπάνω ορισμό του Δευτερεύοντως τότε δεν επιτρέπεται να καθοριστεί ως Παρεμφερές. Το Έγγραφο μπορεί να μην περιέχει Παρεμφερή Κεφάλαια.  Αν το Έγγραφο δεν αναγνωρίζει κάποια Δευτερεύοντα Κεφάλαια τότε δεν υπάρχουν καθόλου.

Τα ``\textbf{Συνοδευτικά κείμενα}'' είναι σίγουρα μικρά περάσματα κειμένου που μπαίνουν σε λίστα, Μπροστινά Συνοδευτικά – Κείμενα ή Πίσω Συνοδευτικά Κείμενα, στη σημείωση που λέει ότι το έγγραφο εκδίδεται υπο την Άδεια. Ένα Μπροστινό Συνοδευτικό Κείμενο μπορεί να είναι το πολύ 5 λέξεις, και ένα Πίσω Συνοδευτικό Κείμενο μπορεί να είναι το πολύ 25 λέξεις.

Ένα ``\textbf{Σαφές}'' αντίγραφο του Εγγράφου σημαίνει ένα αναγνώσιμο αντίγραφο απο το μηχάνημα, παρουσιασμένο σε μια μορφή της οποίας ο προσδιορισμός είναι διαθέσιμος στο ευρύ κοινό, το οποίο είναι κατάλληλο για αναθεώρηση του εγγράφου ευθέως με επεξεργαστές κειμένου ή (για εικόνες που συντάσονται απο pixels) γενικά προγράμματα ζωγραφικής ή (για ζωγραφιές) μερικούς ευρέως διαθέσιμους επεξεργαστές ζωγραφικής, και που είναι κατάλληλο για εισαγωγή σε μορφοποιητές κειμένου ή για αυτόματη μετάφραση σε έναν αριθμό μορφών κατάλληλων για εισαγωγή σε μορφοποιητές κειμένου. Ένα αντίγραφο που έχει δημιουργηθεί σε κάποια άλλη μορφή Σαφή αρχείου του οποίου η διατύπωση, η απουσία διατύπωσης έχει κανονιστεί έτσι ώστε να ματαιώσει ή να αποτρέψει επακόλουθες τροποποιήσεις απο τους αναγνώστες δεν είναι Σαφές. Μια μορφή εικόνας δεν είναι Σαφής αν χρησιμοποιηθεί για ένα μεγάλο κείμενο. Ένα αντιγραφο που δεν είναι Σαφές ονομάζεται ``\textbf{Ασαφές}''.

Παραδείγματα κατάλληλων μορφών για Σαφή αντίγραφα συμπεριλαμβάνουν σκέτα ASCII χωρίς διατύπωση (markup), μορφή TexInfo, LaTex, SGML ή XML χρησιμοποιώντας ένα ευρέως διαθέσιμο DTD, και απλή HTML, PostScript ή PDF σχεδιασμένο για ανθρώπινες τροποποιήσεις. Παραδείγματα απο Σαφή μορφή εικόνας συμπεριλαμβάνουν PNG, XCF και JPG. Ασαφής μορφές περιλαμβάνουν μορφές κλειστού κώδικα που μπορούν να διαβαστούν και να επιμεληθούν μόνο απο αποκλειστικούς επεξεργαστές κειμένου, SGML ή XML για τους οποίους τα DTD και/ή τα εργαλεία επεξεργασίας δεν είναι γενικώς διαθέσιμα, και η παραγόμενη απο το μηχάνημα HTML, PostScript ή PDF που παράγονται απο μερικούς επεξεργαστές κειμένου για παραγόμενα αποτελέσματα μόνο.

H ``\textbf{Σελίδα τίτλου}'' σημαίνει, για ένα τυπωμένο βιβλίο, την ίδια τη σελίδα τίτλου, και επιπρόσθετα όλες τις επακόλουθες σελίδες που χρειάζονται για να κρατήσουν, νομιμοποιήσουν, το υλικό που χρειάζεται αυτή η άδεια για να εμφανιστεί στη σελίδα τίτλων. Για εργασίες σε μορφές που δεν έχουν κάποιον τίτλο σελίδας όπως, “Τίτλος σελίδας” σημαίνει το κείμενο δίπλα στη πιο περίοπτη εμφάνιση του τίτλου της εργασίας, με το κύριο σώμα του κειμένου να προηγείται.

Ο ``\textbf{Εκδότης}'' σημαίνει οποιοδήποτε άτομο ή οντότητα που διανέμει αντίγραφα του Εγγράφου στο κοινό. 

Ένα κεφάλαιο που  ``\textbf{Τιτλοφορείται ΧΥΖ}'' σημαίνει μια ονομασμένη υπομονάδα του Εγγράφου του οποίου ο τίτλος είναι ακριβώς ΧΥΖ ή εμπεριέχει ΧΥΖ σε παρενθέσεις ακολουθούμενο απο κείμενο που μεταφράζει το ΧΥΖ σε μια άλλη γλώσσα (εδώ το ΧΥΖ αναφέρεται σε συγκεκριμένο όνομα κεφαλαίου που αναφέρεται παρακάτω όπως  ``\textbf{Ευχαριστίες}'',``\textbf{Αφιερώσεις}'', ``\textbf{Προσυπογραφές}'', ή ``\textbf{Ιστορικό}''.)  
Για να ``\textbf{Διατηρήσετε τον τίτλο}''
ενός τέτοιου κεφαλαίου όταν τροποποιείται το Έγγραφο  σημαίνει ότι παραμένει ένα κεφάλαιο “Τιτλοφορημένο ΧΥΖ” σύμφνωνα με τον ορισμό.

Το Έγγραφο μπορεί να συμπεριλαμβάνει Αποποιήσεις Εγγύησης δίπλα στη σημείωση που δηλώνει ότι αυτή η Άδεια ισχύει για το Έγγραφο. Αυτές οι Αποποιήσεις Εγγύησης θεωρούνται ότι συμπεριλαμβάνονται απο αναφορά σ'αυτή την Άδεια, αλλά μόνο σαν συναφή άδειες αποποίησης: οποιαδήποτε άλλη εμπλοκή που αυτές οι Αποποιήσεις Άδειας μπορεί να έχουν είναι άκυρη και δεν έχει καμία επίδραση στο νόημα αυτής της Άδειας.


\begin{center}
{\Large\bf 2. ΑΝΤΙΓΡΑΦΗ ΕΠΙ ΛΕΞΗ\par}
\end{center}

Μπορείτε να αντιγράψετε και να διανείμετε το Έγγραφο σε οποιδήποτε μέσο, είτε εμπορικά είτε μη εμπορικά, δεδομένου ότι αυτή η Άδεια, οι σημειώσεις πνευματικών δικαιωμάτων, και η σημείωση άδειας που λέει ότι αυτή η Άδεια ισχύει για το Έγγραφο, αναπαράγονται σε όλα τα αντίγραφα, και ότι δεν προσθέτετε άλλους όρους πέρα απο αυτούς που αναφέρονται στην Άδεια. Δεν μπορείτε να προσθέσετε τεχνικές μετρήσεις για να παρεμποδίσετε ή να ελέγξετε το διάβασμα ή περαιτέρω αντιγραφή των αντιγράφων που κάνετε ή διανέμετε. Παρ'όλα αυτά μπορείτε να δεχτείτε αποζημίωση σε αντάλαγμα για τα αντίγραφα. Αν διανείμετε ένα μεγάλο αριθμό απο αντίγραφα πρέπει επίσης να ακολουθήσετε τους όρους του κεφαλαίου 3.

Μπορείτε επίσης να δανείσετε αντίγραφα, κάτω απο τους ίδιους όρους που περιγράφονται παραπάνω, και μπορείτε να αναδεικνύετε αντίγραφα δημόσια.


\begin{center}
{\Large\bf 3. ΑΝΤΙΓΡΑΦΩΝΤΑΣ ΣΕ ΠΟΣΟΤΗΤΑ\par}
\end{center}


Αν δημοσιεύσετε εκτυπωμένα αντίγραφα (ή αντίγραφα σε μέσα που κοινώς έχουν εκτυπώσει εξώφυλλα) του Εγγράφου, αριθμώντας περισσότερα απο 100, και η σημείωση άδειας του Εγγράφου απαιτεί Κείμενα Εξωφύλλου, πρέπει να συμπεριλάβετε τα αντίγραφα σε εξώφυλλα που φέρουν, καθαρά και νόμιμα, όλα αυτά τα Κείμενα Εξωφύλλου: κείμενα εμπροσθόφυλλου στο μπροστινό μέρος, και κείμενα οπισθόφυλλου πίσω μέρος. Και το εμπροσθόφυλλο και το οπισθόφυλλο πρέπει καθαρά και νόμιμα να αναφέρουν εσάς ως εκδότη αυτών των αντιγράφων. Το εμπροσθόφυλλο πρέπει να παρουσιάζει τον πλήρες τίτλο με όλες τις λέξεις του τίτλου το ίδιο περίοπτες και εμφανείς. Μπορείτε επιπρόσθετα να προσθέσετε άλλο υλικό στα εξώφυλλα. Η αντιγραφή με αλλαγές που περιρίζονται στα εξώφυλλα, όσο διατηρούν ίδιο τον τίτλο του Εγγράφου και πληρούν αυτές τις προϋποθέσεις, μπορούν να θεωρηθούν ως αυτολεξεί αντιγραφή.

Αν τα απαιτούμενα κείμενα για οποιοδήποτε απο τα εξώφυλλα είναι πολύ ογκώδη για να ταιριάξει νομικά, μπορείτε να βάλετε τα πρώτα που βρίσκονται στη λίστα (όσα λογικά χωράνε) στο πραγματικό εξώφυλλο και συνεχίστε με τα υπόλοιπα στις προσαρμοστέες σελίδες.

Αν εκδίδετε ή διανέμετε “Ασαφή” αντίγραφα του Εγγράφου αριθμώντας περισσότερα απο 100, πρέπει να συμπεριλάβετε ένα αναγνώσιμο Διαφανές αντίγραφο μαζί με κάθε Ασαφές, ή να δηλώσετε μέσα ή μαζί με κάθε Ασαφές αντίγραφο μια τοποθεσία δικτύου Η/Υ απο όπου το κοινό που χρησιμοποιεί το δίκτυο να έχει πρόσβαση να κατεβάσει χρησιμοποιώντας τα στάνταρ πρωτόκολλα δικτύου, ένα ολοκληρωμένο “Σαφές” αντίγραφο του Εγγράφου, ελεύθερο απο επιπρόσθετα υλικά. Αν χρησιμοποιήσετε την τελευταία επιλογή, πρέπει να κάνετε συνετά βήματα, όταν ξεκινήσετε να διανέμετε Ασαφή αντίγραφα σε ποσότητα, για να σιγουρέψετε οτι αυτό το Σαφές αντίγραφο θα παραμείνει έτσι προσβάσιμο στην τοποθεσία που έχει δηλωθεί τουλάχιστον για ένα χρόνο μετά την τελευταία φορά που θα διανείμετε ένα Ασαφές αντίγραφο (κατευθείαν ή διαμέσω των πρακτόρων ή των λιανοπωλητών) αυτής της έκδοσης προς το κοινό.

Ζητείται, αλλά δεν απαιτείται, να επικοινωνήσετε με του συγγραφείς του Εγγράφου καλώς πριν αναδιανείμενετε ενα μεγάλο αριθμό απο αντίγραφα, έτσι ώστε να τους δώσετε την ευκαιρία να σας παρέχουν μια ανανεωμένη έκδοση του Εγγράφου.


\begin{center}
{\Large\bf 4. ΤΡΟΠΟΠΟΙΗΣΕΙΣ\par}
\end{center}

Μπορείτε να αντιγράψετε και να διανείμετε μια Τροποιημένη Έκδοση του Εγγράφου υπο του όρους του κεφαλαίου 2 και 3 παραπάνω, δεδομένου ότι θα κυκλοφορήσετε την Τροποιημένη Έκδοση ακριβώς υπο τη συγκεκριμένη Άδεια, με την Τροποποιημένη Έκδοση να γεμίζει το ρόλο του Εγγράφου, αδειοδοτώντας έτσι τη διανομή και την αλλαγή της Τροποποιημένης Έκδοσης σε οποιονδήποτε κατέχει ένα αντίγραφο. Επιπρόσθετα, πρέπει να κάνετε τα παρακάτω στην Τροποποιημένη Έκδοση:

\begin{itemize}
\item[A.] 
   Να χρησιμοποιήσετε στη Σελίδα Τίτλων (και στα εξώφυλλα, αν υπάρχουν) ένα τίτλο ξεχωριστό απο  εκείνο του Εγγράφου, και απο εκείνους προηγούμενων εκδόσεων (που θα έπρεπε, αν υπάρχουν, να μπουν σε λίστα στο κεφάλαιο Ιστορικού του Εγγράφου). Μπορείτε να χρησιμοποιήσετε όπως σε προηγούμενη έκδοση αν ο αρχικός εκδότης αυτής της έδοσης δίνει άδεια.
   
\item[B.]
   Συγκαταριθμήστε στη Σελίδα Τίτλων, ως συγγραφείς, ένα ή περισσότερα άτομα που είναι υπεύθυνα για τη συγγραφή των αλλαγών στην Τροποποιημένη Έκδοση, μαζί με τουλάχιστον πέντε απο τους κύριους συγγραφείς του Εγγράφου (όλους τους κύριους συγγραφείς, αν αυτοί είναι λιγότεροι απο πέντε), εκτός αν σας απελευθερώνουν απο αυτή την προϋπόθεση.
   
\item[Γ.]
   Δηλώστε στη Σελίδα Τίτλων το όνομα του εκδότη της Τροποποιημένης Έκδοσης, ως τον εκδότη.
   
\item[Δ.]
   Διατηρήστε όλες τις σημειώσεις περι πνευματικών δικαιωμάτων του Εγγράφου.
   
\item[E.]
   Προσθέστε μια κατάλληλη σημείωση περί πνευματικών δικαιωμάτων για τις τροποποιήσεις δίπλα στις άλλες σημειώσεις πνευματικών δικαιωμάτων.
   
\item[ΣΤ.]
   Συμπεριλάβετε, αμέσως μετά τις σημειώσεις περί πνευματικών δικαιωμάτων, μια σημείωση άδειας που θα δίνει το ελεύθερο στο κοινό να χρησιμοποιήσει την Τροποποιημένη Έκδοση υπο τους όρους της Άδειας, με τη μορφή που δείχνεται στο Παράρτημα παρακάτω.
   
\item[Ζ.]
   Διατηρήστε σε αυτή την σημείωση άδειας την πλήρη λίστα των Αμετάβλητων Κεφαλαίων και τα απαιτούμενα Κείμενα Εξωφύλλων που δίνονται στη σηεμίωση Άδειας του Εγγράφου.
   
\item[Η.]
   Συμπεριλάβετε ένα μη τροποποιημένο αντίγραφο της Άδειας.
   
\item[Θ.]
   Διατηρήστε το κεφάλαιο με τίτλο “Ιστορικό”, διατηρήστε τον τίτλο του, και προσθέστε του ένα αντικείμενο που να δηλώνει τουλάχιστον τον τίτλο, το έτος, τους καινούργιους συγγραφείς και τον εκδότη της Τροποιημένης Έκδοσης όπως δίνεται στη Σελίδα Τίτλων. Αν δεν υπάρχει καφάλαιο με το όνομα “Ιστορικό” στο Έγγραφο, δημιουργήστε ένα που να αναφέρει τον τίτλο, το έτος, τους συγγραφείς και τον εκδότη του Εγγράφου όπως δίνονται στη Σελίδα Τίτλων, μετά προσθέστε ένα αντικείμενο που θα περιγράφει την Τροποποιημένη Έκδοση όπως δηλώνεται στην προηγούμενη πρόταση. 
   
\item[Ι.]
    Διατηρήστε την τοποθεσία δικτύου, αν υπάρχει, που δίνεται στο Έγγραφο για δημόσια πρόσβαση σε ένα Σαφές αντίγραφο του Εγγράφου, και παρόμοια τις τοποθεσίες δικτύων που δίνονται στο Έγγραφο για προηγούμενες εκδόσεις πάνω στις οποίες βασιζόταν. Μπορεί να τοποθετηθούν στο κεφάλαιο “Ιστορικό”. Μπορείτε να παραλήψετε μια τοποθεσία δικτύου για μια εργασία που είχε εκδοθεί τουλάχιστον τέσσερα χρόνια πριν το Έγγραφο το ίδιο, ή αν ο αρχικός εκδότης της έκδοσης στην οποία αναφέρεται δίνει άδεια.
   
\item[ΙΑ.]
   Για κάθε κεφάλαιο που θα τιτλοφορείται ως “Ευχαριστίες” ή “Αφιερώσεις”, διατηρείστε τον Τίτλο του κεφαλαίου και διατηρήστε στο κεφάλαιο όλη την ουσία και τον τόνο των ευχαριστιών του κάθε συντελεστή και/ή τις αφιερώσεις που δίνοται εκεί.
   
\item[ΙΒ.]
   Διατηρείστε όλα τα παρεμφερή Κεφάλαια του Εγγράφου, χωρίς να αλλάξετε το κείμενό τους και τους τίτλους τους. Οι αριθμοί του Κεφαλαίου ή τα αντίστοιχα δεν θεωρούνται μέρος των τίτλων του κεφαλαίου.
   
\item[ΙΓ.]
   Διαγράψτε οποιοδήποτε κεφάλαιο φέρει τίτλο “Προσυπογραφή”. Ένα τέτοιο κεφάλαιο μπορεί να μην συμπεριληφθεί στην Τροποποιημένη Έκδοση.
   
\item[ΙΔ.]
   Μην επανατιτλοφορείτε κάποιο υπάρχων κεφάλαιο με Τίτλο “Προσυπογραφή” ή με το να έρχεται σε σύγκρουση με οποιοδήποτε Αμετάβλητο Κεφάλαιο.
   
\item[ΙΕ.]
    Διατηρείστε οποιαδήποτε Αποποίηση Εγγύησης.
\end{itemize}

Αν στην Τροποποιημένη Έκδοση συμπεριλαμβάνονται νέα εισαγωγικά κείμενα ή παραρτήματα που χαρακτηρίζονται ως Δευτερεύοντα Κεφάλαια και δεν περιέχουν υλικό αντιγραμένο απο το Έγγραφο, μπορείτε απο επιλογή σας να καθορίσετε μερικά ή όλα απο αυτα τα κεφάλαια ως παρεμφερή. Για να το κάνετε αυτό, προσθέστε τους τίτλους τους στη λίστα των Αμετάβλητων Κεφαλαίων στη σημείωση άδειας της Τροποποιημένης Έκδοσης. Αυτοί οι Τίτλοι πρέπει να είναι ξεχωριστοί απο άλλους τίτλους κεφαλαίων.

Μπορείτε να προσθέσετε ένα κεφάλαιο με τίτλο “Προσυπογραφές”, δεδομένου ότι δεν περιέχει τίποτα άλλο απο προσυπογραφές της Τροποποιημένης Έκδοσης σας απο διάφορους συμμετέχοντες – για παράδειγμα, δηλώσεις ανασκόπησης ή ότι το κείμενο έχει εγκριθεί απο έναν οργανισμό ως ο έγκυρος ορισμός ενός στάνταρ.

Μπορείτε να προσθέσετε ένα χωρίο μέχρι πέντε λέξεις ως ένα κείμενο εμπροσθόφυλλου και ένα χωρίο ως 25 λέξεις ως κείμενο οπισθόφυλλου, στο τέλος της λίστας των Κειμένων Εξωφύλλων στην τροποποιημένη έκδοση. Μόνο ένα χωρίο κειμένου Εμπροσθόφυλλου και ένα κείμενο Οπισθόφυλλου μπορούν να προστεθούν απο (ή μέσω διευθετήσεων που θα γίνουν) ένα οποιοδήποτε άτομο. Αν το Έγγραφο συμπεριλαμβάνει ήδη ένα κείμενο εξωφύλλου για το ίδιο εξώφυλλο, που προηγουμένως έχει προστεθεί απο εσάς ή απο διευθέτηση που έγινε απο το ίδιο άτομο εκ του οποίου δράτε, μην προσθέσετε άλλο; αλλά μπορείτε να αντικαταστήσετε το παλιό, με κατηγορηματική άδεια απο τον προηγούμενο εκδότη.

Ο συγγραφέας (-είς) και οι εκδότης (-ες) αυτού του Εγγράφου δεν δίνουν το ελεύθερο με αυτή την Άδεια μα χρησιμοποιήσουν τα ονόματα τους δημόσια για να υποστηρίξουν ή να υπονοήσουν προσυπογραφή οποιασδήποτε Τροποποιημένης Έκδοσης.


\begin{center}
{\Large\bf 5. ΣΥΝΔΥΑΖΟΝΤΑΣ ΕΓΓΡΑΦΑ\par}
\end{center}


Μπορείτε να συνδυάσετε το Έγγραφο με άλλα έγγραφα που εκδίδονται υπο την αυτή την Άδεια, υπο τους όρους που καθορίστικαν στο κεφάλαιο 4 για τροποιημένες εκδόσεις, δεδομένου ότι συμπεριλαμβάνετε στο συνδυασμό όλα τα Παρεμφερή Κεφάλαια όλων των αρχικών εγγράφων, μη τροποιημένα και συγκαταριθμήστε τα όλα ως Παρεμφερή Κεφάλαια της συνδυασμένης σας δουλειάς στη σημείωση άδειας και δεδομένου ότι διατηρείτε όλα τα Παρεμφερή Κεφάλαια.

Η συνδυασμένη δουλειά χρειάζεται μόνο να περιέχει ένα αντίγραφο της Άδειας και τα πολλαπλά πανομοιότυπα Παρεμφερή Κεφάλαια μπορούν να αντικατασταθούν με ένα μοναδικό αντίγραφο. Αν υπάρχουν πολλαπλά Παρεμφερή Κεφάλαια με το ίδιο όνομα αλλα διαφορετικό περιεχόμενο, κάντε τον τίλτο κάθε τέτοιου κεφαλαίου μοναδικό προσθέτοντας στο τέλος, σε παρενθέσεις, το όνομα του αρχικού συγγραφέα ή εκδότη του κεφαλαίου αν είναι γνωστό, ή αλλιώς ένα μοναδικό νούμερο. Κάντε την ίδια ρύθμιση στους τίτλους κεφαλαίου στη λίστα των Αμετάβλητων Κεφαλαίων στη σημείωση άδειας τη συνδυασμένης δουλειάς. 

Στο συνδυασμό, πρέπει να συνδυάσετε οποιοδήποτε κεφάλαιο έχει τίτλο “Ιστορικό” στα διάφορα αρχικά κείμενα, σχηματίζοντας ένα κεφάλαιο με τίτλο “Ιστορικό”; παρόμοια συνδυάστε οποιοδήξποτε κεφάλαιο φέρνει τίτλο “Ευχαριστίες” και οποιοδήποτε κεφάλαιο φέρνει τίτλο “Αφιερώσεις”. Πρέπει να σβήσετε όλα τα κεφάλαια με τίτλο “Προσυπογραφές”. 

\begin{center}
{\Large\bf 6. ΣΥΛΛΟΓΕΣ ΕΓΓΡΑΦΩΝ\par}
\end{center}

Μπορείτε να κάνετε μια συλλογή ή οποία να αποτελείται απο το Έγγραφο και άλλα έγγραφα που έχουν κυκλοφορήσει υπο αυτή την Άδεια και αντικαταστήστε τα μοναδικά αντίγραφα της Άδειας στα διάφορα έγγραφα με ένα μοναδικό αντίγραφο που συμπεριλαμβάνεται στη συλλογή, δεδομένου ότι ακολουθείτε τους κανόνες αυτής της Άδειας για αυτολεξεί αντιγραφή καθενός απο τα έγγραφα.

Μπορείτε να αποσπάσετε ένα μοναδικό απο μια τέτοια συλλογή και να το διανείμετε ξεχωριστά υπο αυτή την Άδεια, δεδομένου ότι εισάγετε ένα αντίγραφο αυτής της Άδειας στο αποσπασμένο έγγραφο, και ακολουθείτε αυτή την Άδεια απ'όλες τις απόψεις συμπεριλαμβάνοντας την αυτολεξεί αντιγραφή του Εγγράφου.


\begin{center}
{\Large\bf 7. ΣΥΣΣΩΜΑΤΣΗ ΜΕ ΑΝΕΞΑΡΤΗΤΕΣ ΕΡΓΑΣΙΕΣ\par}
\end{center}


Ένας συνδυασμός του Εγγράφου ή των παραγώγων του με άλλα ξεχωριστά και ανεξάρτητα έγγραφα ή εργασίες, μέσα ή πάνω σε ένα όγκο ενός μέσου αποθήκευσης ή διανομής, ονομάζεται “σύνολο” αν τα πνευματικά διακαιώματα που προκύπτουν απο το συνδυασμό δεν χρησιμοποιούνται για να περιορίσουν τα νομικά διακαιώματα των χρηστών του συνδυασμού πέρα απ'ότι επιτρέπει η συγκεκριμένη εργασία. Όταν το Έγγραφο συμπεριλαμβάνεται σε ένα σύνολο, αυτή η Άδεια δεν ισχύει στις άλλες εργασίες του συνόλου οι οποίες δεν είναι απο μόνες τους παράγωγα εργασίας του Εγγράφου.

Αν η απαίτηση του κειμένου του Εξωφύλλου του κεφαλαίου 3 είναι εφαρμόσιμη σε αυτά τα αντίγραφα του Εγγράφου, τότε αν το Έγγραφο είναι λιγότερο απο το μισό ολόκληρου του συνόλου, τα κείμενα Εξωφύλου του Εγγράφου μπορούν να τοποθετηθούν σε εξώφυλλα που κατηγοριοποιούν το Έγγραφο μέσα στο σύνολο, ή τα ηλεκτρονικά αντίστοιχα εξωφύλλων αν το Έγγραφο είναι σε ηλεκτρονική μορφή. Αλλιώς πρέπει να εμφανιστούν σε εκτυπωμένα εξώφυλλα που κατηγοριοποιούν όλο το σύνολο.


\begin{center}
{\Large\bf 8. ΜΕΤΑΦΡΑΣΗ\par}
\end{center}


Η μετάφραση θεωρείται ένα είδος τροποποίησης, οπότε μπορείτε να διανείμετε μεταφράσεις του Εγγράφου υπο τους όρους του κεφαλαίου 4. Αντικαθιστώντας Παρεμφερή Κεφάλαια με μεταφράσεις απαιτεί ειδική άδεια απο τους κατόχους των πνευματικών διακαιωμάτων, αλλά μπορείτε να συμπεριλάβετε μεταφράσεις μερικών ή όλων των Αμετάβλητων Κεφαλαίων επιπρόσθετα στις αρχικές εκδόσεις αυτών των Αμετάβλητων Κεφαλαίων. Μπορείτε να συμπεριλάβετε μια μετάφραση αυτής της Άδειας, και όλες τις σημειώσεις άδειας στο Έγγραφο, και οποιαδήποτε Αποποίηση Άδειας, δεδομένου ότι θα συμπεριλάβετε την αυθεντική Αγγλική έκδοση αυτής της Άδειας και τις αρχικές εκδόσεις αυτών των σημειώσεων και αποποιήσεων. Σε περιπτωση διαφωνίας μεταξύ της μετάφρασης και της αρχικής έκδοσης αυτής της Άδειας ή μιας σημείωσης ή μιας αποποίησης, η αρχική έκδοση υπερισχύει.

Αν ένα κεφάλαιο στο έγγραφο έχει τίτλο "Ευχαριστίες”, “Αφιερώσεις” ή “Ιστορικό”, η απαίτηση (κεφάλαιο 4) για Διατήρη του Τίτλου (κεφάλαιο 1) τυπικά θα απαιτήσει την αλλαγή του πραγματικού τίτλου.


\begin{center}
{\Large\bf 9. ΟΛΟΚΛΗΡΩΣΗ\par}
\end{center}


Δεν μπορείτε να αντιγράψετε, να τροποιήσετε, να αδειοδοτήσετε ή να διανείμετε το Έγγραφο παρά μόνο όπως ρητώς προβλέπεται υπο αυτή την Άδεια. Οποιαδήποτε άλλη απόπειρα για αντιγραφή, τροποποίηση, αδειοδότηση ή διανομή είναι άκυρη και αυτόματα θα τερματίσει τα δικαιώματα σας υπο αυτή την Άδεια.

Παρόλα αυτά, αν σταματήσετε τις παραβάσεις αυτής της Άδειας, τότε η άδειά σας απο ένα συγκεκριμένο ιδιοκτήτη πνευματικής ιδιοκτησίας αποκαθίσταται (α) προσωρινά, εκτός εάν και μέχρι ο ιδιοκτήτης των πνευματικών δικαιωμάτων κατηγορηματικά και τελικά τερματίσει την άδειά σας, και (β) μόνιμα, αν ο ιδιοκτήτης πνευματικών διακαιωμάτων δεν σας ενημερώσει για την παράβαση με κάποια λογικά μέσα πριν περάσουν 60 μέρες απο μετά τη διακοπή.

Επιπλέον, η άδειά σας απο ένα συγκεκριμένο ιδιοκτήτη πνευματικών διακαιωμάτων αποκαθίσταται μόνιμα αν ο ιδιοκτήτης των πνευματικών διακαιωμάτων σας ειδοποιήσει για την παραβίαση με κάποια λογικά μέσα, είναι η πρώτη φορά που λαμβάνετε σημείωση παραβίασης αυτής της άδεια (για οποιαδήποτε εργασία) απο αυτό τον ιδιοκτήτη πνευματικών διακιωμάτων, και αποκαθιστάτε την παραβίαση μέσα σε 30 μέρες απο τη στιγμή που θα λαβετε τη σημείωση.

Τερματισμός των διακιωμάτων σας υπο αυτό το κεφάλαιο δεν τερματίζει την άδεια ατόμων που έχουν λάβει αντίγραφα ή διακαιώματα απο εσάς υπο αυτή την Άδεια. Αν τα διακαιώματά σας έχουν τερματιστεί και δεν αποκατασταθούν μόνιμα, η απόδειξη ενός αντιγράφου μερικού ή όλου του υλικού δεν σας δίνει κανένα διακαίμα για χρήση του.


\begin{center}
{\Large\bf 10. ΜΕΛΛΟΝΤΙΚΕΣ ΑΝΑΘΕΩΡΗΣΕΙΣ ΑΥΤΗΣ ΤΗΣ ΑΔΕΙΑΣ\par}
\end{center}


Το Ιδρυμα Ελεύθερου Λογισμικού (Free Software Foundation) ίσως εκδόσει νέες, αναθεωρημένες εκδόσεις της GNU Free Documentation License ανα διαστήματα. Τέτοιες νέες εκδόσεις θα είναι παρεμφερείς με την παρούσα έκδοση, αλλά μπορεί να διαφέρουν σε λεπτομέρεια όσον αφορά τα νέα προβλήματα και θέματα. Βλέπετε
http://www.gnu.org/copyleft/.

Σε κάθε έκδοση αυτής της άδειας δίνεται ένας χαρακτηριστικός αριθμός έκδοσης. Αν το Έγγραφο καθορίζει ότι μια συγκεκριμένη αριθμημένη έκδοση αυτής της Άδειας “ή κάποια μετέπειτα έκδοση”  ισχύει για αυτή, έχετε την επιλογή να ακολουθήσετε τους όρους και τις προϋποθέσεις είτε αυτής της καθορισμένης έκδοσης ή οποιασδήποτε μετέπειτα έκδοσης έχει εκδοθεί ή οποιασδήποτε μετέπειτα έκδοσης έχει εκδοθεί (όχι ως προσχέδιο) απο το Ίδρυμα Ελεύθερου Λογισμικού. Αν το Έγγραφο δεν καθορίζει έναν αριθμό έκδοσης αυτής της Άδειας, μπορείτε να διαλέξετε οποιαδήποτε έκδοση δημοσιεύθηκε ποτέ (όχι ως προσχέδιο) απο το Ίδρυμα Ελεύθερου Λογισμικού. Αν το Έγγραφο καθορίζει ότι μια πληρεξουσιότητα μπορεί να αποφασίσει ποιές μελλοντικές εκδόσεις αυτής της Άδειας μπορούν να χρησιμοποιηθούν, η δημόσια δήλωση αποδοχής μιας έκδοσης αυτού του πληρεξουσίου σας εξουσιοδοτεί μόνιμα να διαλεξετε αυτή την έκδοση για το Έγγραφο.


\begin{center}
{\Large\bf 11. ΕΠΑΝΕΞΟΥΣΙΟΔΟΤΗΣΗ\par}
\end{center}


Το “Massive Multiauthor Collaboration Site” (ή “MMC Site”) σημαίνει οποιονδήποτε διακομηστή (server) του Παγκόσμιου Ιστού που εκδίδει εργασίες με πνευματικά δικαιώματα και παρέχει επίσης περίπτες μονάδες για οποιονδήποτε θέλει να επεξεργαστεί αυτές τις εργασίες. Ένα δημόσιο wiki όπου οποιοσδήποτε μπορεί να επεξεργαστεί είναι ένα τέτοιο είναι ένα παράδειγμα ενός τέτοιου διακομηστή. Ένα “Massive Multiauthor Collaboration” (ή “MMC”) που εμπεριέχεται στην ιστοσελίδα σημαίνει οποιοδήποτε σετ εργασιών με πνευματικά διακαιώματα που κατα συνέπεια αναρτώνται στην ιστοσελίδα του MMC. 

“CC-BY-SA” σημαίνει την άδεια Creative Commons Attribution-Share Alike 3.0 δημοσιευμένη απο την Creative Commons Corporation, μια μη κερδοσκοπική επιχείρηση με κύρια μέρος το San Francisco, California, όπως επίσης μελλοντικές copyleft εκδόσεις αυτής της άδειας δημοσιευμένες απο τον ίδιο οργανισμό. 

“Ενσωμάτωση” σημαίνει η έκδοση ή επανέκδοση του Εγγράφου, ολόκληρου ή μέρους του, ως μέρος άλλου Εγγράφου.

Ένα MMC είναι ``δικαιούχο για επανεξουσιοδότηση'' αν έχει αδειοδοτηθεί υπο αυτή την Άδεια, και αν όλα δουλεύουν, και που είχαν πρώτα εκδοθεί υπο αυτή την Άδεια κάπου αλλού εκτός απο αυτό το MMC και επομένως ενσωματώθηκαν εξ ολοκλήρου ή κατά ένα τμήμα μέσα στο MMC, (1) δεν είχε κείμενα εξωφύλλου ή Παρεμφερή κεφάλαια και (2) επομένως ενσωματώθηκε πριν απο την 1η Νοεμβρίου του 2008.

Ο χειριστής μιας ιστοσελίδας MMC, μπορεί να επανεκδώσει ένα MMC που περιλαμβάνεται στην ιστοσελίδα υπο το CC-BY-SA στην ίδια ιστοσελίδα οποιαδήποτε χρονική στιγμή πριν απο την 1η Αυγούστου 2009, δεδομένου ότι το MMC είναι νόμιμο για επανέκδοση.


\begin{center}
{\Large\bf ΠΑΡΑΡΤΗΜΑ: Πως να χρησιμοποιήσετε αυτή την Άδεια για τα έγγραφα σας\par}
\end{center}

Για να χρησιμοποιήσετε αυτή την Άδεια σε ένα κείμενο που έχετε γράψει, συμπεριλάβετε ένα αντίγραφο της Άδειας στο έγγραφο και βάλτε τις παρακάτω σημειώσεις άδειας και πνευματικών διακαιωμάτων ακριβώς μετά τη σελίδα τίτλων:

\bigskip
\begin{quote}
    Copyright \copyright{}  ΕΤΟΣ ΤΟ ΟΝΟΜΑ ΣΑΣ.
    Δίνεται άδεια για αντιγραφή, διανομή και/ή τροποποίηση αυτού του εγγράφου υπο τους όρους του GNU Free Documentation License, Version 1.3 ή και μετέπειτα έκδοση δημοσιευμένη απο το Ίδρυμα Ελεύθερου Λογισμικού; χωρίς Παρεμφερή Κεφάλαια, χωρίς κείμενα Εμπροσθόφυλλου και χωρίς κείμενα Οπισθόφυλλου. Ένα αντίγραφο της άδειας συμπεριλαμβάνεται στο κεφάλαιο με τίτλο “GNU Free Documentation License”. 
\end{quote}
\bigskip
    
Αν έχετε Παρεμφερή Κεφάλαια, κείμενα Εμπροσθόφυλλου και Οπισθόφυλλου, αντικαταστήστε την γραμμή ``με \dots\ Κείμενα.'' με την παρακάτω:

\bigskip
\begin{quote}
    με τα Παρεμφερή Κεφάλαια να είναι ΣΥΓΚΑΤΑΡΙΘΜΗΣΤΕ ΤΑ ΟΝΟΜΑΤΑ ΤΟΥΣ, με τα κείμενα Εμπροσθόφυλλου να είναι ΣΥΓΚΑΤΑΡΙΘΜΗΣΤΕ και με τα κείμενα Οπισθόφυλλου να είναι ΣΥΓΚΑΤΑΡΙΘΜΗΣΤΕ.
\end{quote}
\bigskip
    
Αν έχετε Παρεμφερή Κεφάλαια χωρίς Κείμενα Εξωφύλλων, ή μερικούς άλλυς συνδιασμούς των τριών συγχωνεύστε αυτές τις δύο εναλακτικές για να βολέψετε την κατάσταση.

Αν το έγγραφο σας εμπεριέχει σημαντικά παραδείγματα κώδικα προγραμματισμού, προτείνουμε να διαθέσετε αυτά τα παραδείγματα παράλληλα υπο την επιλογή της άδειας ελεύθερου λογισμικού, όπως η GNU General Public License, έτσι ώστε να επιτρέψετε τη χρήση τους στο ελεύθερο λογισμικό.

