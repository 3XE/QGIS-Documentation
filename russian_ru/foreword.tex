%  !TeX  root  =  user_guide.tex
\mainmatter
\pagestyle{scrheadings}
\addchap{Предисловие}\label{label_forward}
\pagenumbering{arabic}
\setcounter{page}{1}

% when the revision of a section has been finalized,
% comment out the following line:
% \updatedisclaimer

Добро пожаловать в удивительный мир географических информационных систем
(ГИС)!

Quantum~GIS (QGIS) является ГИС с открытым исходным кодом. Работа
над QGIS была начата в мае 2002~года, а в июне того же года "--- создан
проект на площадке SourceForge. Мы много работали, чтобы сделать
программное обеспечение ГИС (которое традиционно является дорогим
проприетарным ПО) доступным любому, кто имеет
доступ к персональному компьютеру. В настоящее время QGIS работает на
большинстве платформ: Unix, Windows, и OS~X. QGIS разработана с
использованием инструментария Qt (\url{http://qt.nokia.com}) и языка
программирования C++. Это означает, что QGIS легка в использовании, имеет
приятный и простой графический интерфейс.

QGIS стремится быть легкой в использовании ГИС, предоставляя общую
функциональность. Первоначальная цель заключалась в облегчении
просмотра геоданных и QGIS достигла той стадии в своем развитии, когда
многие используют ее в своих ежедневных задачах просмотра.
QGIS поддерживает множество растровых и векторных форматов данных, а
поддержка новых форматов реализуется с помощью модулей.

QGIS выпускается на условиях лицензии GNU General Public License (GPL).
Разработка QGIS под этой лицензией означает, что вы можете просмотреть и
изменить исходный код, и гарантирует, что вы, наш счастливый
пользователь, всегда будете иметь доступ к программному обуспечению ГИС,
которое является бесплатным и может свободно адаптироваться. Вы должны
были получить полную копию лицензии с вашей копией QGIS, лицензию также
можете найти в Приложении~\ref{gpl_appendix}.

\begin{Tip}\caption{\textsc{Актуальная версия документации}}\index{документация}
Актуальную версию данного документа всегда можно найти на странице
\url{http://download.osgeo.org/qgis/doc/manual/}, или в разделе
документации на веб-сайте QGIS \url{http://www.qgis.org/en/documentation}
\end{Tip}

%%% TODO: add link to the manual
Русскоязычную версию руководства, созданную в рамках коллективного
проекта GIS-Lab, можно найти по адресу:
\url{http://gis-lab.info/docs/qgis/manual17/qgis-1.7.0_user_guide_ru.pdf}.
На данный момент доступен перевод версии 1.7 руководства.

\addsec{Возможности}\label{label_majfeat}

\qg позволяет использовать большое количество распространенных ГИС функций,
обеспечиваемых встроенными инструментами и модулями. Первое представление
можно получить из краткого резюме ниже, где функции разбиты на шесть
категорий.

\minisec{Просмотр данных}

Можно просматривать и накладывать друг на друга векторные и растровые
данные в различных форматах и проекциях без преобразования во внутренний
или общий формат. Поддерживаются следующие основные форматы:

\begin{itemize}[label=--]
\item пространственные таблицы PostGIS, векторные форматы, поддерживаемые
установленной библиотекой OGR, включая shape-файлы ESRI, MapInfo, SDTS
(Spatial Data Transfer Standard), GML (Geography Markup Language) и многие
другие.
\item Форматы растров и графики, поддерживаемые
библиотекой GDAL (Geospatial Data Abstraction Library), такие, как
GeoTIFF, Erdas IMG, ArcInfo ASCII Grid, JPEG, PNG и многие другие.
\item базы данных SpatiaLite (см. Раздел~\ref{label_spatialite})
\item растровый и векторный форматы GRASS (область/набор данных),
см. Раздел~\ref{sec:grass}.
\item пространственные данные, публикуемые в сети Интернет с помощью
OGC-совместимых (Open Geospatial Consortium) сервисов Web Map Service
(WMS) или Web Feature Service (WFS), см. Раздел~\ref{working_with_ogc}.
\item данные OpenStreetMap (OSM), см. Раздел~\ref{plugins_osm}.
\end{itemize}

\minisec{Исследование данных и компоновка карт}

С помощью удобного графического интерфейса можно создавать карты и
исследовать пространственные данные. Графический интерфейс включает в
себя множество полезных инструментов,например:

\begin{itemize}[label=--]
\item перепроецирование <<на лету>>
\item компоновщик карт
\item панель обзора
\item пространственные закладки
\item определение/выборка объектов
\item редактирование/просмотр/поиск атрибутов
\item подписывание объектов
\item изменение символики векторных и растровых слоев
\item добавление слоя координатной сетки "--- теперь средствами расширения
fTools
\item добавление к макету карты стрелки на север, линейки масштаба
и знака авторского права
\item сохранение и загрузка проектов
\end{itemize}

\minisec{Управление данными: создание, редактирование и экспорт}

В QGIS можно создавать и редактировать векторные данные, а также
экспортировать их в разные форматы. Чтоб иметь возможность редактировать
и экпортировать в другие форматы растровые данные, необходимо
сначала импортировать их в GRASS. QGIS предоставляет следующие возможности
работы с данными, в частности:

\begin{itemize}[label=--]
\item инструменты оцифровки для форматов, поддерживаемых библиотекой OGR,
и векторных слоев GRASS
\item создание и редактирование shape-файлов и векторных слоев GRASS
\item геокодирование изображений с помощью модуля пространственной
привязки
\item инструменты GPS для импорта и экспорта данных в формате GPX,
преобразования прочих форматов GPS в формат GPX или скачивание/загрузка
непосредственно в прибор GPS (в Linux usb был добавлен в список
устройств GPS)
\item визуализация и редактирование данных OpenStreetMap
\item создание слоёв PostGIS из shape-файлов с помощью плагина SPIT
\item обработка слоёв PostGIS
\item управление атрибутами векторных данных с помощью новой таблицы
атрибутов (см. Раздел~\ref{sec:attribute table}) или модуля Table Manager
\item сохранение снимков экрана как изображений с пространственной
привязкой
\end{itemize}

\minisec{Анализ данных}

Вы можете анализировать векторные пространственные данные в PostgreSQL/PostGIS
и других форматах, поддерживаемых OGR, используя модуль fTools, написанный на
языке программирования Python. В настоящее время QGIS предоставляет возможность
использовать инструменты анализа, выборки, геопроцессинга, управления
геометрией и базами данных. Также можно использовать интегрированные
инструменты GRASS, которые включают в себя функциональность более чем
400 модулей GRASS (см. Раздел~\ref{sec:grass}).

\minisec{Публикация карт в сети Интернет}

QGIS может использоваться для экспорта данных в map-файл и публикации
его в сети Интернет, используя установленный веб-сервер Mapserver.
QGIS может использоваться как клиент WMS/WFS и как сервер WMS.

\minisec{Расширение функциональности QGIS с помощью модулей расширения}

QGIS может быть адаптирован к особым потребностям с помощью расширяемой
архитектуры модулей. QGIS предоставляет библиотеки, которые могут
использоваться для создания модулей. Можно создавать отдельные
приложения, используя языки программирования C++ или Python.

\minisec{Основные модули}

\begin{enumerate}
\item Добавить слой из текста с разделителями (загружает и выводит
текстовые файлы, содержащие координаты x,y)
\item Захват координат (получает координаты мыши в различных системах
координат)
\item Оформление (знак авторского права, стрелка на север, масштабная
линейка)
\item Наложение диаграмм (наложение диаграмм на векторные слои)
\item Смещение точек (активация режима отрисовки, который делает возможным
сдвиг точек с одинаковыми координатами)
\item Преобразователь Dxf2Shp (преобразование файлов DXF в shape-файлы)
\item Инструменты GPS (загрузка и импорт данных GPS)
\item GRASS (поддержка ГИС GRASS)
\item Инструменты GDAL (интеграция инструментов GDAL в QGIS)
\item Привязка растров GDAL (географическая привязка растров)
\item Модуль интерполяции (интерполяция векторных данных)
\item Экспорт в Mapserver (экспорт проекта QGIS в map-файл Mapserver)
\item Оффлайновое редактирование (оффлайновое редактирование слоёв и
синхронизация с базами данных)
\item Модуль OpenStreetMap (просмотр и редактирование данных
OpenStreetMap)
\item Доступ к данным Oracle Spatial GeoRaster
\item Установщик модулей Python (загрузка и установка модулей QGIS)
\item Морфометрический анализ (морфометрический анализ растровых слоев)
\item Road graph (поиск кратчайшего маршрута)
\item SPIT (инструмент импорта shape-файлов в PostgreSQL/PostGIS)
\item SQL Anywhere (работа с векторными слоями в БД SQL Anywhere)
\item Пространственные запросы (пространственные запросы для векторных слоёв)
\item Модуль WFS (загрузка слоёв WFS)
\item eVIS (инструмент визуализации событий "--- показ изображений, связанных
с векторными объектами)
\item fTools (инструменты для управления векторными данными и их анализа)
\item Консоль Python (доступ к среде разработки QGIS из самой программы)

\end{enumerate}

\minisec{Внешние модули Python}

QGIS предлагает постоянно растущее число модулей Python, которые
разрабатываются сообществом. Они находятся в официальном
репозитории PyQGIS, и могут быть легко установлены с помощью Установщика
модулей Python (см. Раздел~\ref{sec:plugins}).

\subsubsection{Что нового в версии \CURRENT}

Имейте ввиду, что этот выпуск является <<нестабильным>>. Это значит, что
помимо новых возможностей в нём, по сравнению с QGIS 1.0.x и QGIS 1.6.0,
расширен программный интерфейс. Мы рекомендуем использовать именно эту
версию вместо предыдущих.

Этот выпуск содержит свыше 277 исправлений, а также и множество новых
возможностей и улучшений.

\minisec{Символика, подписи и диаграммы}

\begin{itemize}[label=--]
\item Новая символика используется по умолчанию
\item Для размещения диаграмм используется тот же алгоритм, что и для
размещения подписей в labeling-ng
\item Экспорт и импорт стилей (новая символика)
\item Подписи для правил при использовании отрисовки по правилам
(rule-based renderer)
\item Смещение по X и Y для символьных маркеров
\item Линейный маркер
\begin{itemize}[label=--]
\item Возможность выводить маркер на центральной точке линии
\item Отрисовка маркеров только на первой или последней вершине линии
\item Возможность выводить маркер на каждой вершине линии
\end{itemize}
\item Заливка
\begin{itemize}[label=--]
\item Возможность поворота заливки в формате SVG
\item Cлой <<отрисовка центроидов>> для заливки полигонов
\item Использование слоёв из линейных знаков для отрисовки контуров полигонов
\end{itemize}
\item Подписи
\begin{itemize}[label=--]
\item Возможность указания отступа в единицах карты
\item Новые инструменты для интерактивного перемещения, вращения и изменения
подписей
\end{itemize}
\end{itemize}

\minisec{Новые инструменты}

\begin{itemize}[label=--]
\item Графический интерфейс для gdaldem.
\item Калькулятор полей с функциями \$x, \$y и \$perimeter.
\item Инструмент преобразования линий в полигоны
\item Инструмент построения диаграмм Вороного
\end{itemize}

\minisec{Пользовательский интерфейс}

\begin{itemize}[label=--]
\item Добавлен диалог обработки отсутствующих слоёв проекта
\item Увеличение до группы слоёв
\item Диалог <<Совет дня>> при запуске программы
\item Улучшена организация меню, добавлено меню <<База данных>>
\item Возможность показывать количество объектов для классов легенды
\item Множественные исправления и улучшения интерфейса
\end{itemize}

\minisec{Управление системами координат}

\begin{itemize}[label=--]
\item Отображение активной системы координат в строке состояния
\item Возможность назначить систему координат слоя для всего проекта
\item Возможность выбора системы координат по умолчанию для новых проектов
\item Возможность изменения системы координат для нескольких слоёв
\item В диалоге выбора систем координат по умолчанию предлагается последний
выбор пользователя
\end{itemize}

\minisec{Работа с растрами}

\begin{itemize}[label=--]
\item Операции AND и OR в калькуляторе растров
\item Преобразование проекции растра <<на лету>>
\item Улучшенная реализация растровых провайдеров
\item Панель инструментов <<Растр>> с функциями растяжения гистограммы
\end{itemize}

\minisec{Источники данных}

\begin{itemize}[label=--]
\item Новый источник данных SQLAnywhere
\item Возможность объединения таблиц
\item Обновления форм редактирования
\begin{itemize}[label=--]
\item Настраиваемое представление значения NULL
\item Исправлено открытие формы редактирования из таблицы атрибутов
\item Поддержка значения NULL в карте значений (выпадающий список)
\item При загрузке карты значений из слоя используются фактические имена слоёв
\item Поддержка выражений в формах редактирования: строчные поля с префиксом
<<expr\_>> считаются выражениями. Их значение интерпретируется как выражение
калькулятора полей и заменяется вычисленным значением
\end{itemize}
\item Поддержка поиска значений NULL в таблице атрибутов
\item Редактирование атрибутов
\begin{itemize}[label=--]
\item Улучшены возможности редактирования в таблице атрибутов (добавление и
объектов, изменение атрибутов)
\item Поддержка объектов без геометрии
\item Исправлены отмена и возврат операций изменения атрибутов
\end{itemize}
\item Множественные улучшения работы с атрибутами
\begin{itemize}[label=--]
\item Возможность повторного использования предыдущих значений атрибутов
для создаваемых объектов
\item Возможность объединения и присваивания значений атрибутов группам объектов
\end{itemize}
\item Возможность сохранения слоёв без атрибутов
\end{itemize}

\minisec{Разработчику}

\begin{itemize}[label=--]
\item Вызовы диалога атрибутов переработаны с использованием QgsFeatureAttribute.
\item Добавлен сигнал QgsVectorLayer::featureAdded
\item Добавлены функции работы с меню <<Слой>>
\item Добавлен параметр путей поиска для двоичных модулей (для активации
требуется перезапуск)
\item Новая функция проверки геометрии в fTools на основе QgsGeometry.validateGeometry.
В новой функции увеличена скорость работы, расширены сообщения об ошибках
и добавлена возможность отображения ошибок на карте.
\end{itemize}

\minisec{Сервер WMS}

\begin{itemize}[label=--]
\item Указание характеристик сервера в свойствах проекта (вместо отдельного файла wms\_metadata.xml)
\item Поддержка печати через wms-запрос GetPrint
\end{itemize}

\minisec{Модули}

\begin{itemize}[label=--]
\item Поддержка значков в Менеджере модулей
\item Удалён модуль быстрой печати (используйте модуль easyprint из репозитория)
\item Удалён модуль преобразования форматов (используйте пункт контекстного
меню <<Сохранить как>>)
\end{itemize}

\minisec{Печать}

\begin{itemize}[label=--]
\item Возможность отмены операций при работе с макетом
\end{itemize}

\newpage
