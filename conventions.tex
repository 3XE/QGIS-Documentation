%  !TeX  root  =  user_guide.tex
\addchap{Conventions}\label{label_conventions}

This section describes a collection of uniform styles throughout the manual. The conventions used in this manual are as follows:

\addsec{GUI Conventions}

The GUI convention styles are intended to mimic the appearance of the GUI. In general, the objective is to use the non-hover appearance, so a user can visually scan the GUI to find something that looks like the instruction in the manual.

\begin{itemize}[label=--,itemsep=5pt]
\item Menu Options: \mainmenuopt{Layer} \arrow
\dropmenuopttwo{mActionAddRasterLayer}{Add a Raster Layer}

or

\mainmenuopt{Settings} \arrow
\dropmenuopt{Toolbars} \arrow \dropmenucheck{Digitizing}
\item Tool: \toolbtntwo{mActionAddRasterLayer}{Add a Raster Layer}
\item Botton : \button{Save as Default}
\item Dialog Box Title: \dialog{Layer Properties}
\item Tab: \tab{General}

\item Toolbox : \toolboxtwo{nviz}{nviz - Open 3D-View in NVIZ}
\item Checkbox: \checkbox{Render}
\item Radio Button:  \radiobuttonon{Postgis SRID} \radiobuttonoff{EPSG ID}
\item Select a Number: \selectnumber{Hue}{60}
\item Select a String: \selectstring{Outline style}{---Solid Line}
\item Browse for a File: \browsebutton
\item Select a Color: \selectcolor{Outline color}{yellow}
\item Slider: \slider{Transparency}{0}{20mm}
\item Input Text: \inputtext{Display Name}{lakes.shp}
\end{itemize}
A shadow indicates a clickable GUI component.

\addsec{Text or Keyboard Conventions}

The manual also includes styles related to text, keyboard commands and coding to indicate different entities, such as classes, or methods. They don't correspond to any actual appearance.

\begin{itemize}[label=--]
%Use for all urls. Otherwise, it is not clickable in the document.
\item Hyperlinks: \url{http://qgis.org}
%\item Single Keystroke: press \keystroke{p}
\item Keystroke Combinations: press \keystroke{Ctrl+B}, meaning press and hold the Ctrl key and then press the B key.
\item Name of a File: \filename{lakes.shp}
%\item Name of a Field: \fieldname{NAMES}
\item Name of a Class: \classname{NewLayer}
\item Method: \method{classFactory}
\item Server: \server{myhost.de}
%\item SQL Table: \sqltable{example needed here}
\item User Text: \usertext{qgis ---help}
\end{itemize}

Code is indicated by a fixed-width font:
\begin{verbatim}
PROJCS["NAD_1927_Albers",
  GEOGCS["GCS_North_American_1927",
\end{verbatim}

\addsec{Platform-specific instructions}

GUI sequences and small amounts of text can be formatted inline: Click \{\nix{}\win{File} \osx{QGIS}\} \arrow Quit to close QGIS.

This indicates that on Linux, Unix and Windows platforms, click the File menu option first, then Quit from the dropdown menu, while on Macintosh OSX platforms, click the \qg menu option first, then Quit from the dropdown menu. Larger amounts of text may be formatted as a list:

\begin{itemize}
\item \nix{do this;}
\item \win{do that;}
\item \osx{do something else.}
\end{itemize}

or as paragraphs.

\nix{} \osx{} Do this and this and this. Then do this and this and this and this and this and this and this and this and this.

\win{}Do that. Then do that and that and that and that and that and that and that and that and that and that and that and that and that and that and that.

Screenshots that appear throughout the user guide have been created on different platforms; the platform is indicated by the platform-specific iconsat the end of the figure caption.
