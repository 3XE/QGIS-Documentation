%  !TeX  root  =  user_guide.tex
\addchap{本书约定}\label{label_conventions}

本节介绍了整个用户指南遵循的统一风格。本书中遵循以下约定:

\addsec{用户界面约定}

约定用户界面样式的目的在于模仿用户界面外观。一般来说,其目的是利用非动态外观让用户可以浏览一些像指南中介绍的东西。

\begin{itemize}[label=--,itemsep=5pt]
\item 菜单选项: \mainmenuopt{图层} \arrow
\dropmenuopttwo{mActionAddRasterLayer}{添加栅格图层}

或者

\mainmenuopt{设置} \arrow
\dropmenuopt{工具条} \arrow \dropmenucheck{数字化}
\item 工具: \toolbtntwo{mActionAddRasterLayer}{添加栅格图层}
\item 按钮: \button{存为默认}
\item 对话框名: \dialog{图层属性}
\item 标签页: \tab{通用}

\item 工具箱: \toolboxtwo{nviz}{nviz - 在NVIZ中打开三维视图}
\item 多选框: \checkbox{渲染}
\item 单选框:  \radiobuttonon{Postgis SRID} \radiobuttonoff{EPSG 号}
\item 选择数字: \selectnumber{色调}{60}
\item 选择名称: \selectstring{边线样式}{---实线}
\item 浏览文件: \browsebutton
\item 选择颜色: \selectcolor{边线颜色}{yellow}
\item 滑块: \slider{透明度}{0}{20毫米}
\item 输入文字: \inputtext{显示名称}{lakes.shp}
\end{itemize}
阴影表明该部分界面元素可以单击。

\addsec{文本或键盘约定}

该指南还包括文本样式、键盘命令样式和编码样式(用来表示类、方法等不同实体),它们没有实际的外观显示。

\begin{itemize}[label=--]
%Use for all urls. Otherwise, it is not clickable in the document.
\item 超链接: \url{http://qgis.org}
%\item Single Keystroke: press \keystroke{p}
\item 组合键: 按 \keystroke{Ctrl+B}, 意思是按住Ctrl键不放,然后按B键。
\item 文件名: \filename{lakes.shp}
%\item Name of a Field: \fieldname{NAMES}
\item 类名: \classname{NewLayer}
\item 方法: \method{classFactory}
\item 服务器: \server{myhost.de}
%\item SQL Table: \sqltable{example needed here}
\item 用户操作说明: \usertext{qgis ---help}
\end{itemize}

代码使用固定宽度字体显示:
\begin{verbatim}
PROJCS["NAD_1927_Albers",
  GEOGCS["GCS_North_American_1927",
\end{verbatim}

\addsec{平台相关的说明}

界面顺序和少量文字可以直接写成一行: 单击 \{\nix{}\win{文件} \osx{QGIS}\} \arrow 退出来退出QGIS。
这表明,在Linux,Unix和Windows系统上,单击“文件”菜单选项,再从下拉菜单中单击“退出”,而在Macintosh OSX系统上,单击“QGIS”菜单选项,再从下拉菜单中选择“退出”。字数较多的文本可能需要列表显示:

\begin{itemize}
\item \nix{如此;}
\item \win{如彼;}
\item \osx{再如彼;}
\end{itemize}

或者图形化显示。

\nix{} \osx{} 如此如此如此,然后如此如此如此。

\win{} 如彼如彼,然后如彼如彼如彼如彼。

整个用户指北中根据需要对不同系统平台进行截图;系统类型在截图标题中用不同的图表区分。