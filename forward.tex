% vim: set textwidth=78 autoindent:

\section{Forward}\label{label_forward}
\pagenumbering{arabic}
\setcounter{page}{1}

Welcome to the wonderful world of Geographical Information Systems (GIS)!
Quantum GIS (QGIS) is an Open Source Geographic Information System. The project
was born in May of 2002 and was established as a project on SourceForge in June
of the same year. We've worked hard to make GIS software (which is traditionaly
expensive commerical software) a viable prospect for anyone with basic access
to a Personal Computer. QGIS currently runs on most Unix platforms, Windows, and
OS X. QGIS is developed using the Qt toolkit (\url{http://www.trolltech.com})
and C++. This means that QGIS feels snappy to use and has a pleasing, easy to
use graphical user interface. 

QGIS aims to be an easy to use GIS, providing common functions and features.
The initial goal was to provide a GIS data viewer. QGIS has reached that point
in its evolution and is being used by many for their daily GIS data viewing
needs. QGIS supports a number of raster and vector data formats, with new
support easily added using the plugin architecture (see Appendix
\ref{appdx_data_formats} for a full list of currenly supported data formats).
QGIS is released under the GNU General Public License (GPL). Developing QGIS 
under this license means that you can inspect and modify the source code
and guarantees that you, our happy user will always have access to a GIS
program that is free of cost and can be freely modified. You should have
received a full copy of the license with your copy of QGIS, and you also
find it as Appendix \ref{gpl_appendix}.  

\begin{quote}
\begin{center}
\textbf{Note:} The latest version of this document can always be found at \newline
http://qgis.org/docs/userguide.pdf 
\end{center}
\end{quote}

\subsection{Features}\label{label_majfeat}

QGIS has many common GIS features and functions. The major features
are listed below, devided into Core Features and Plugins. \\

\textbf{Core Features}

\begin{itemize}
\item Raster and vector support by the OGR library
\item Support for spatially enabled PostgreSQL tables using PostGIS
\item GRASS integration, including view, edit, and analysis
\item Digitizing GRASS and OGR/Shapefile
\item Map Composer
\item OGC support
\item Overview panel
\item Spatial bookmarks
\item Identify/Select features
\item Edit/View/Search attributes
\item Feature labeling
\item On the fly projection
\item Save and restore projects
\item Export to Mapserver map file
\item Change vector and raster symbology 
\item Extensible plugin architecture
\end{itemize}

\textbf{Plugins}

\begin{itemize}
\item Add WFS Layer
\item Add Delimited Text Layer
\item Decorations (Copyright Label, North Arrow and Scale bar)
\item Georeferencer
\item GPS Tools
\item GRASS
\item Graticule Creator
\item PostgreSQL Geoprocessing functions
\item SPIT Shapefile to PostgreSQL/PostGIS Import Tool
\item Python Console
\item openModeller
\end{itemize}

\subsection{Whats New in 0.9}\label{label_whatsnew}

Version 0.9.0 brought some very interesting new features to you.

\begin{itemize}
\item Python language bindings to write plugins in Python and to create GIS 
enabled applications in Python that use the QGIS libraries
\item Removed automake build system - QGIS now needs CMake for compilation
\item Many new GRASS modules added to the GRASS toolbox
\item Map Composer updates
\item Fix for 2.5D shapefiles
\item Improvements to the Georeferencer
\item Localization support extended to 26 languages    
\end{itemize}

QGIS \CURRENT concentrates on stabilization and feature enhancement.

\begin{itemize}
\item 66 bugfixes and feature improvements 
\item New window arrangement feature for the Georeferencer
\item New locale tab in the options dialog
\item Download progress information for WMS and WFS data
\item More GRASS modules added to the GRASS toolbox
\end{itemize}
